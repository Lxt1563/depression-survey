% !TEX root = tnnls_depression_survey.tex

\ifx\allfiles\undefined
    \input{tnnls_prefix}
\fi

\section{Auxiliary depression diagnosis method based on Brain Imaging}
\label{sec_approach}

The structure and function of brain are abnormal in depressed patients.
To a certain extent, they inhibit the speed of thinking of patients, seriously causing cognitive damage, and even mental illness.
In addition, depression can atrophy the hippocampus of the patient's brain, change neurotransmitters, and cause chronic inflammation, leading to memory loss, slow thinking, and fatigue, which greatly reduces the energy level and motivation of the patient and makes it difficult for them to socialize.
These findings have found a new breakthrough point for auxiliary diagnosis of depression.

As shown in Fig \ref{fNIRS}(a), brain images are acquired by scanning the subject's brain with various precision instruments, and there are significant differences in brain volume and structure among different people, as well as a large amount of interference and noise in the acquisition data due to the slight head movement of the subject or the scanning equipment during the acquisition process, so image pre-processing is needed to eliminate individual differences and noise interference.
Among them, image normalization includes head movement correction, brain tissue separation, and segmentation alignment, this step can remove individual differences and head movement interference.
Image denoising is mainly used to remove the noise generated by the external environment and equipment, and the common methods include Gaussian filtering, band-pass filtering, wavelet transform, etc.
In the feature extraction process, linear models, wavelet packet decomposition and other methods are usually applied to extract time and frequency domain features, or the features of each brain region are extracted according to standard brain templates, and then redundant and irrelevant features are removed using feature selection algorithms, and the effective features are fused for subsequent prediction.

Brain imaging refers to techniques that employ an interaction between brain tissue and various forms of energy (e.g., electromagnetic or particle radiation), rather than physical incision, to capture positional data about the structure and function of the brain.
Brain images can be divided into MRI, fNIRS, Computed Tomography, and Postron Emission Computed Tomography, Magnetic Resonance Spectroscopy etc. according to different imaging technologies.
These brain images have several advantages, such as safety, painless, non-invasive, which are important tools for depression diagnosis.
As other types of images are used too infrequently, we mainly focus on MRI-based and fNIRS auxiliary diagnosis of depression.


\begin{figure}[tbp]
	\centering
	\subfloat[]{
		\label{fig_hard_case1}\includegraphics[width=0.5\linewidth]{figures/depression/fNIRS3.png}
	}
	\subfloat[]{
		\label{fig_hard_case2}\includegraphics[width=0.5\linewidth]{figures/depression/fNIRS4.png}
	}\\
	\subfloat[]{
		\label{fig_hard_case1}\includegraphics[width=0.5\linewidth]{figures/depression/sMRI.png}
	}
	\subfloat[]{
		\label{fig_hard_case2}\includegraphics[width=0.5\linewidth]{figures/depression/fMRI.png}
	}
	\caption{
    (a) Participant instrumented and sitting upright while collecting brain imaging data~\cite{2020Classifying}.
    (b) PFC activation sensitivity map of fNIRS probe~\cite{2020Classifying}.
    (c) Three View of sMRI images~\cite{marek2011parkinson}
    (d) Three View of fMRI images~\cite{marek2011parkinson}
	}
	\label{fNIRS}
\end{figure}


\subsection{MRI}
%\label{sec_fquality}
%%%%%%%%%%%%%%%%%%%%%%%%%%%%%%%%%%%%%%%

MRI is a non-invasive imaging technology that produces three dimensional detailed anatomical images. 
It is often used for disease detection, diagnosis, and treatment monitoring.
MRI is a examination technology based on the principle that atomic nuclei with magnetic distance can produce transitions between energy levels under the action of a magnetic field.
Physicians are able to tell the difference between various types of tissues based on these magnetic properties, and help to check the energy state and cerebral blood flow of the patient's brain. 

MRI can not only describe the changes in the volume or structure of brain tissue, but also reflect the changes in neural activity in the brain. It provides a great value in the auxiliary diagnosis of depression.
It has been found that depressed patients show paradoxical changes in the hippocampal volume, gray matter volume, habenula volume, etc., which usually show varying degrees of reduction~\cite{Santos2018Global,2011Volume,2005Male,2020Subthreshold,2013Habenula,J2013P}.

In MRI techniques, images obtained with different scanning parameters include structural MRI (sMRI) and functional MRI (fMRI) as shown in Fig \ref{fNIRS}(c)(d).
The sMRI images can clearly show the anatomical structure of the brain with high spatial resolution and can be used to objectively reflect the structural morphological changes within the brain~\cite{Santos2018Global,2011Volume,2013Habenula,J2013P,2013Evaluation,P2020Radiological,2019Depression}.
fMRI uses magnetic resonance imaging to measure changes in hemodynamics caused by neuronal activity. 
It can observe the changes of the brain through the small changes in the magnetic resonance signal caused by the oxygenation state of the brain, thereby revealing the relationship between brain activity and thinking.
And most of the researches based on MRI auxiliary diagnosis of depression are conducted using fMRI.


%The detection of depression from sMRI scans usually includes processes such as image acquisition and preprocessing, feature extraction and selection, and classification. Machine learning methods have also been widely used, such as One Rule (OneR), SVM, Information Gain (IG) and ReliefF, etc~\cite{2013Evaluation}~\cite{P2020Radiological}.
%Deep learning methods have brought out remarkable results in medical imaging analysis, such as Mousavian et al.~\cite{2019Depression} found that the pre-trained VGG 16 and fine- tuning produced good results.


According to different experimental conditions, fMRI is mainly divided into Resting State fMRI and Task State fMRI.
Task State fMRI refers to the process of collecting data, subjects need to perform specific tasks, such as exercise, cognitive activities.
Task State fMRI is model driven and requires careful experiment design. The quality of the model is closely related to the experiment design.
%The method based on TS-fMRI usually use Generalize Linear Model to extract linear features, and then applying statistical tests for analysis.
In contrast, Resting State fMRI is the fMRI image when the brain does not perform specific cognitive tasks, and keeps quiet, relaxed and awake~\cite{2018Disrupted}. 
Resting State fMRI does not depend on the experimental design. It also has some physiological significance to detect the blood oxygen level dependent signal of subjects in resting state.
In addition, Resting State fMRI images have the ability to reveal low-frequency blood oxygenation level dependent signal fluctuations, which can be used to monitor the abnormal spontaneous neuronal activity caused by mental diseases.
%Therefore, RS-fMRI is also more used to assist in the diagnosis of depression.

%Many researchers have built machine learning classification models to predict the accuracy of depression~\cite{2013Identifying,2019Intrinsic,2019Resting}.
%Zeng et al.~\cite{2014Unsupervised} built an unsupervised machine learning model for MDD.
%Bhaumik et al.~\cite{2016Multivariate} extracted the features of the left posterior cingulate cortex and the right dorsolateral prefrontal cortex and input it to the SVM classifier for depression recognition.
%Yan et al.~\cite{2020Quantitative} exploited the hidden information embedded in dynamic functional connectivity (DFC) and developed an accurate and objective image-based diagnosis system for MDD.
%To improve the generalization ability of classifiers, multicenter studies~\cite{2020Generalizable} have been conducted and cover deep learning algorithms.
%Zhao et al.~\cite{2020Functional} established a generative adversarial network depression classification model based on functional brain network connections in a multicenter sample. Jun rt al.~\cite{2020Identifying} distinguished drug-naïve MDD patients from healthy controls using Graph Convolutional Networks (GCNs). 


%Murrough utilized functional magnetic resonance imaging (fMRI) and two separate emotion perception tasks to examine the neural effects of ketamine in patients with treatment-resistant major depressive disorder(TRD), found that TRD patients had reduced neural responses to frontal faces in the right caudate nucleus compared to healthy volunteers~\cite{2015Regulation}.
%Johnston described an instrumental loss-avoidance and win-gain reinforcement learning functional magnetic resonance imaging study with 40 patients with highly treatment-resistant major depressive disorder and never-depressed controls, found that the predictive diagnostic accuracy of patients with abnormal striatal activity was as high as 97\%, while the predictive diagnostic accuracy of abnormal hippocampal activity was 84\%~\cite{2015Failure}.
%Current research combining machine learning with TS fMRI is mainly related to emotional face stimulation, speech and music listening tasks, and this method enables individual-level analysis and reduces subjective misjudgments.
%Machine learning classifiers have achieved good results in related research, such as Gaussian Process Classifiers(GPC), RF, SVM, etc~\cite{2013What}~\cite{2015Toward}.
%Because deep learning has the advantages of reducing human intervention, deep analysis and feature extraction, selection and classification in the same optimal deep structure. Gui trained a Deep Learning Network (DLN) model with fMRI data of listening to positive and negative music~\cite{2019The1}.


\subsubsection{fMRI auxiliary depression diagnosis based on traditional machine learning}
%%%%%%%%%%%%%%%%%%%%%%%%%%%%%%%%%%%%%%%

The method based on fMRI usually use Generalize Linear Model to extract linear features, and then applying statistical tests for analysis
However, more nonlinear features such as Functional Connectivity, Degree Centrality, Regional Homogeneity, and Amplitude of Low-frequency Fluctuation are used for analysis.

(1) Functional Cnnectivity refers to statistical dependence between time series of electro-physiological activity and oxygenated blood levels in distinct regions of the brain.
Abnormalities in connectivity between functional areas of the brain can reflect changes by mental illness.

(2) Degree Centrality reflects the importance of nodes in information transmission in the brain functional network.
Degree Centrality is an effective index, which has been widely used to find changes in resting-state functional networks in mental illness.

(3) Regional Homogeneity measures brain activity which evaluates the similarity or synchronization between the time series of a given voxel and its nearest neighbors.
This metric has been commonly applied to study the association between healthy or disease conditions.

(4) Amplitude of Low-frequency Fluctuation measures the total power of a given time course within a typical frequency range, which is typically 0.01-0.08Hz.
Amplitude of Low-frequency Fluctuation can characterize the energy intensity of brain activity over a period of time and effectively delineate the potential pathophysiological mechanisms of diseases.

These features can better describe the characteristics of different brain regions or voxel regions, and are important features for describing Resting State fMRI images~\cite {lord2012changes,2013Identifying,2016Multivariate,wang2017depression,mousavian2020depression}.

Multivariable Pattern Analysis is also used in the research of individual level diagnosis and classification of mental diseases.
This method can judge whether an unknown type of subject sample has certain diseases by modeling multivariable data of a predetermined type.
On this basis, Gu et al.~\cite {8575800} use multivariable pattern analysis to carry out the Resting State fMRI based auxiliary diagnosis of depression.

However, the common problem based on traditional machine learning methods is that the feature dimension is too high, and the variables of the input model greatly exceed the number of training samples, resulting in over fitting.
Therefore, many researchers choose the least absolute shrink and selection operator to capture the key features, and further optimize them on this basis, using group least absolute shrink and selection operator, sparse group least absolute shrink and selection operator and standard least absolute shrink and selection operator for analysis~\cite{2015Toward}~\cite{ 2020Generalizable }.

\subsubsection{fMRI auxiliary depression diagnosis based on deep learning}
%%%%%%%%%%%%%%%%%%%%%%%%%%%%%%%%%%%%%%%
Although the traditional machine learning methods still occupy the mainstream research position in the medical image field, it need to manually extract features, which requires researchers to have some prior knowledge in related fields, and there is also the problem of removing redundant features, making the feature extraction process complex and burdensome.
Deep learning technology provides more choices for medical image analysis by building neural networks from the main learning features.

Many researchers use CNN to process independent two-dimensional slices to learn effective features in fMRI, thereby better mining the strong local spatial correlation in the image.~\cite{mousavian2020depression}.
Gui et al.~\cite{2019The1} propose a deep learning network with multibranch and local residual feedback, for four different types of Task State fMRI imaging data produced by depressed patients and control people under the music task condition.
Recently, graph based methods are popular in medical applications because they can adapt to the complex pairwise similarity of imaging/nonimaging features between subjects.
Jun et al.~\cite{2020Identifying} use spectral Graph Convolutional Networks based on population graph to successfully integrate effective connectivity features and nonimaging demographic features.

However, MRI data is 3 Dimensions (3D) or 4 Dimensions, which means that it is high-dimensional, including not only spatial information but also time information.
Mousavian et al.~\cite {mousavian2020depression} use 3D-CNN to capture 3D spatial information, or Spatial-temporal CNN to extract spatial and temporal information at the same time.
In addition, in order to understand the changes of the brain over a period of time, LSTM is also trained to capture the changes of Resting State fMRI over time.
Mousavian et al.~\cite{mousavian2020depression} also use a hybrid model combining CNN and LSTM to capture the spatial and temporal information of data.
CNN works for retrieving spatial features of the data and extract the active areas, and LSTM retrieves temporal features and models the blood flow.

The lack of training samples is a serious problem in the application of deep learning methods in medical imaging. Therefore, some researchers use Generative Adversarial Networks(GAN) to solve the problem of over fitting, but the effect is not ideal, and further research is needed~\cite {2020Functional}.

%\subsection{Others}
%%%%%%%%%%%%%%%%%%%%%%%%%%%%%%%%%%%%%%%
%Brain metabolic imaging provides an accurate understanding of nerve cell activity and metabolic changes under normal conditions and disease states, as well as the metabolic situation of the cerebral cortex under different physiological conditions of stimulation and thinking activity. Brain metabolic imaging such as PET and MRS can visualize the metabolic activity and various physiological or pathological metabolic changes in human brain and reflect them in the form of images.

%PET uses the decay law or distribution characteristics of substances related to nuclear radiation, such as glucose, proteins, nucleic acids, etc., in the research object to obtain detailed information to reflect their metabolic activities to achieve the purpose of diagnosis.
%PET is mainly based on the annihilation effect of positrons and electrons and can be used to explore characteristic pathological states and indicators in depressed patients. By investigating the correlation between changes in metabolic function in MDD brain regions and depressive symptoms, it was found that compared to normal healthy subjects, depressed patients usually show such phenomena as decreased regional cerebral blood flow values and standardized glucose uptake values in the prefrontal lobe, increased total distribution of translocated proteins, and increased inflammatory markers~\cite{2017Functional,2017Elevated,li2018microglial,2014Cerebral}.

%MRS uses nuclear chemical shifts to study molecular structure and can be used to detect metabolite concentrations in brain regions.
%MRS can detect biochemical changes in the injured tissue of the body by magnetic resonance hydrogen spectroscopy, which can effectively observe the metabolism of the injured area and has some clinical value for early diagnosis of the depression.
%It was found that N-acetylaspartate (NAA) levels were significantly lower in the left hippocampus, glutamine and glutamate (Glx) levels were significantly increased in the right hippocampus, and choline complex (Cho) and creatine (Cr) levels were significantly higher in depressed patients compared to controls, and correlated with the severity of depression in patients~\cite{2018Early,20181H,2013}.


%\subsection{PET}
%\label{sec_fquality}
%%%%%%%%%%%%%%%%%%%%%%%%%%%%%%%%%%%%%%%
%PET uses the decay law or distribution characteristics of substances related to nuclear radiation, such as glucose, proteins, nucleic acids, etc., in the research object to obtain detailed information to reflect their metabolic activities to achieve the purpose of diagnosis.

%By examining the correlation between changes in cerebral blood perfusion and glucose metabolic function in the prefrontal lobe and depressive symptoms, Fu found that regional cerebral blood flow values and standardized glucose uptake values in the prefrontal lobe were reduced in depressed patients compared to normal healthy subjects~\cite{2017Functional}.
%Holmes believes that major depression is associated with elevated peripheral inflammatory markers. By comparing the utilization of transporters in the anterior cingulate cortex, frontal cortex, and insulation of patients with depression and healthy controls, the results found that transporter availability was significantly increased in glial cells of the anterior cingulate cortex during episodes in patients with MDD~\cite{2017Elevated}.
%Li explored the relationship between microglia-mediated processes and the pathophysiology of MDD by examining the distribution of  translocator protein total distribution volume $(TSPO \quad V_T)$, the results showed that the increased total distribution of translocated proteins was associated with decreased cognitive function in patients with MDD~\cite{li2018microglial}.

%\subsection{MRS}
%\label{sec_fquality}
%%%%%%%%%%%%%%%%%%%%%%%%%%%%%%%%%%%%%%%
%MRS uses nuclear chemical shifts to study molecular structure and can be used to detect metabolite concentrations in brain regions.

%N-acetyl aspartate(NAA) measures were interrogated through examining their relationship to previously documented ELS markers-cerebrospinal fluid (CSF) corticotropin -releasing factor (CRF) concentrations, hippocampal volume, body mass and behavioral timidity. Coplan's study found that compared with the control group, the NAA level in the left hippocampus of depression patients was significantly reduced, and the Glutamine and Glutamate(Glx) level in the right hippocampus was significantly increased~\cite{2018Early}.

%%%%%%%%%%%%%%%%%%%%%%%%%%%%%%%%%%%%%%

\subsection{fNIRS}
\label{sec_fquality}
%%%%%%%%%%%%%%%%%%%%%%%%%%%%%%%%%%%%%%%
fNIRS uses the scattering of blood to near-infrared light to obtain the changes of oxygenated hemoglobin (HbO), deoxygenated hemoglobin (HbR) and total hemoglobin (HbT) during brain activity, which reflects the activities of neuronal and brain cognitive function, and is a useful tool to auxiliary diagnosis of depression
%fNIRS utilizes the scattering properties of blood to near-infrared light to obtain changes in oxyhemoglobin and deoxyhemoglobin during brain activity.
%It is an important tool for the study of brain function imaging. It can be used to compare the differences in brain function between healthy and mentally ill patients, and to achieve non-invasive detection of brain function.
~\cite{2018Reduced,2019Brain,2014Relationship,2016Correlation,2019The}.

\subsubsection{fNIRS auxiliary depression diagnosis based on traditional machine learning}
%%%%%%%%%%%%%%%%%%%%%%%%%%%%%%%%%%%%%%%
The research on fNIRS based auxiliary depression diagnosis is still mainly based on traditional machine learning algorithms~\cite{2019Machine}~\cite{2017Machine}, which heavily rely on manually extracted features.
Linear features are usually used in fNIRS analysis, such as mean, peak, valley, average, variance and other feature indicators of the blood oxygen response function (HbO, HbR, HbT).
These are the features of task related cortical regions extracted by measuring the process of motor imagination task, visual task, cognitive task, etc., and recognized by machine learning classifiers such as SVM, RF, One Rule, eXtreme Gradient Boosting~\cite{2015Automatic,2017Automatic,2020Classifying}.
%In recent years, machine learning techniques have shown great potential in the diagnosis and treatment of mental health disorders including depression~\cite{2019Machine}~\cite{2017Machine}. 
%The most common machine learning algorithms for predictive classification such as Support Vector Machines (SVM), Decision Trees (DT) and naïve bayes,etc, have been applied to fNIRS data to classify Major Depressive Disorder (MDD) patients and healthy individuals. All of these methods have yielded good results in depression identification~\cite{2017Automatic}. 
%Hong et al.~\cite{2015Automatic} designed a SVM based classifier to classify the features extracted from fNIRS.
%There are also some integrated machine learning methods that have been applied in depression recognition, such as Random Forest (RF) and Extreme Gradient Boosting (XGBoost),etc. Zhu et al.~\cite{2020Classifying} extracted ten features from HbO signals, from each channel served as inputs to XGBoost and RF algorithms developed for each block and combination of successive blocks.

Secondly, some researchers applied the statistical parameter mapping method based on the General Linear Model to the task of auxiliary depression diagnosis.
In this method, the linear combination of fNIRS data and regression quantity reflecting the design of stimulation scheme is fitted. Therefore, General Linear Model has higher statistical effectiveness than the average value, because it considers the whole fNIRS time series and takes advantage of the higher sampling rate recorded by fNIRS~\cite{2015Automatic2}.

%Fu recruited 43 patients with bipolar depression to assess changes in frontal region oxyhemoglobin (oxy-Hb) levels during the TOL task and VFT using a 41-channel NIRS system. The results showed that the level of oxyhemoglobin in the prefrontal cortex was significantly lower in patients with bipolar depression~\cite{2018Reduced}.
%By studying the association between brain mitochondrial dysfunction and mood disorders, Holper~\cite{2019Brain} found that brain cytochrome cyclooxygenase (COX) activity was decreased in patients with depression, and was inversely proportional to the severity of depression.
%Liu confirmed that hypoactivation in prefrontal cortex of MDD patients with anxious and obsessive-compulsive symptoms~\cite{2014Relationship}.
%Kawano found that oxy-Hb in the frontal lobe was negatively correlated with depression detection scale scores on NIRS~\cite{2016Correlation}.
In addition, other researchers are committed to extracting features from fNIRS frequency domain signals, and using Fourier transform, wavelet transform and other methods to analyze the signal spectrum to extract effective recognition features~\cite {2015Automatic}.

\subsubsection{fNIRS auxiliary depression diagnosis based on deep learning}
%%%%%%%%%%%%%%%%%%%%%%%%%%%%%%%%%%%%%%%
%Traditional machine learning classification methods require a priori feature preprocessing to train the model, extensive reliance on manual feature extraction and selection. 
%By using deep Learning methods like Artificial Neural Network (ANN), Deep Neural Networks (DNNs) and Long Short-Term Memory (LSTM), etc, records can be fed directly to the algorithm for training, avoiding the need for feature selection. As a result, more and more researchers have applied deep learning algorithms to multimodal fNIRS classification tasks~\cite{2022Deep},and have also achieved high accuracy in depression recognition and diagnosis~\cite{2021Depression}.
%Ma et al.\cite{2020DISTINGUISHING} was able to distinguish bipolar depression from major depressive disorder in adults during a verbal fluency task by using an LSTM~.
%Chao et al.~\cite{2021fNIRS} used a cascade forward neural network to classify depression and achieved excellent results.
The emergence of AlexNet confirmed the dominance of deep learning in computer vision, and also provided more possibilities for the application of deep learning methods in fNIRS auxiliary depression diagnosis~\cite{2022Deep}. 
This CNN method recognizes the original blood data, manually extracted HbO, HbR, HbT features and channel correlation as inputs~\cite{2021Depression}.

The RNN method makes the artificial neural network have a strong ability of time series signal classification and prediction, so it has a wide range of applications in the field of natural language processing. Because fNIRS data has significant time series characteristics, the method based on RNN is gradually applied to this field. Among them, LSTM is the most representative, which can solve the long-term dependence problem of general RNN, and can effectively analyze and predict the time series signals in fNIRS data~\cite{2021fNIRS}.

In addition, Ma et al.~\cite{2020DISTINGUISHING } combine CNN and LSTM to increase the network's ability to extract hidden information from fNIRS data, making the model more comprehensive and stable.


Brain imaging technology has revealed abnormalities in brain structure, brain function and brain metabolism in patients with depression, providing new ideas for early diagnosis and optimization of treatment plans.


\begin{table*}
\centering
\caption{Experimental results based on brain imaging}
\label{tab3}
\setlength\tabcolsep{2mm}{
\begin{tabular}{c|c|c|c|c}
\hline
\textbf{Type}          & \textbf{Paper}                                                                                          & \textbf{Dataset(D+C)}                                                                                                     & \textbf{Methods}                                                                                                                                       & \textbf{Metrics(Accuracy/\%)}                                                                                                        \\
\hline

\multirow{24}{*}{fMRI} & Lord et al. ~\cite{lord2012changes}                                             & 43 (21+22)                                                                                                                 & SVM                                                                                                                                                    & 99.00                                                                                                                                \\
\cline{2-5}
                       & Wei et al.~\cite{2013Identifying}                                                 & 40 (20+20)                                                                                                                 & SVM                                                                                                                                                    & 90.00                                                                                                                                 \\
\cline{2-5}
                       & Zeng et al. ~\cite{2014Unsupervised}                                             & 53 (24+29)                                                                                                                 & Unsupervised ML                                                                                                                                        & 92.50                                                                                                                                \\
                       \cline{2-5}
                       & Yu et al.~\cite{2015Toward}                                                        & 62 (31+31)                                                                                                                 & \makecell[c]{sLASSO\\ gLASSO  \\ sgLASSO  \\ SVM}                                                              & \makecell[c]{83.82  \\ 91.95 \\ 90.08   \\ 94.68}                                              \\
                      \cline{2-5}
                       & Bhaumik~\cite{2016Multivariate}                                                   & 67 (38+29)                                                                                                                 & SVM                                                                                                                                                    & 76.10                                                                                                                                \\
\cline{2-5}
                       & Gu et al. ~\cite{8575800}                                                        & 92 (46+46)                                                                                                                 & Multivariate Pattern Analysis  & 90.22                                                                                                                                \\
\cline{2-5}
                       & Wang et al.~\cite{wang2017depression}                                            & 60 (31+29)                                                                                                                 & SVM                                                                                                                                                    & 95.00                                                                                                                                \\
\cline{2-5}
                       & Gui et al. ~\cite{2019The1}                                                      & 39 (19+20)                                                                                                                 & DNN                                                                                                                                                    & 94.68                                                                                                                                \\
                       \cline{2-5}
                       & Mousavian et al.~\cite{mousavian2020depression}                                   & NKI online dataset 279 (38+241)                                                                                             & \begin{tabular}[c]{@{}c@{}}CNN-LSTM \\ Spatial-temporal CNN \\ 3D CNN   \\ 3D VGG-16  \\ Functional Connectivity Network \\ Functional Connectivity Network\_woT-test  \\ Functional Connectivity Network\_CanICA\end{tabular} & \begin{tabular}[c]{@{}c@{}}100.00  \\ 100.00  \\ 90.00   \\ 100.00   \\ 94.00   \\ 86.00   \\ 87.00\end{tabular} \\
                       \cline{2-5}
                       & Mousavian et al.~\cite{9680063}                                  & NKI online dataset 279 (38+241)                                                                                             & Cubes-Atlas                                                                                                                                            & 93.00                                                                                                                                \\
\cline{2-5} 
                       & Yan et al.~\cite{2020Quantitative}                                                & 99 (43+56)                                                                                                                 & Dynamic Functional Connectivity+SVM                                                                                                                                                & 99.13 (AUC)                                                                                                                           \\
 \cline{2-5}
                       & Zhao et al. ~\cite{2020Functional}                                               & 555 (269+286)                                                                                                              & Functional Network Connectivity+GAN                                                                                                                                                & 70.10                                                                                                                                \\
 \cline{2-5}
                       & Jun et al.~\cite{2020Identifying}                                                  & 73 (29+44)                                                                                                                 & Graph Convolutional Networks & 74.10                                                                                                                                \\
                      \cline{2-5}
                       & Yamashita et al.~\cite{ 2020Generalizable}                                     & \begin{tabular}[c]{@{}c@{}}discovery dataset 713 (149+564)   \\ independent validation dataset 521 (236+285)\end{tabular} & LASSO                                                                                                                                                  & approximately 70.00                                                                                                                 
\\
\hline                                                                                                                                                                                                                                                                                                                                                                                                                                                                                                                                                                                                                                                       
\multirow{6}{*}{fNIRS} & Song et al. ~\cite{2015Automatic}                                                & 138 (108+30)                                                                                                               & SVM                                                                                                                                                    & 86.77                                                                                                                                \\
\cline{2-5} 
                       & Song et al.~\cite{2015Automatic2}                                                 & 138 (108+30)                                                                                                               & SVM                                                                                                                                                    & 89.71                                                                                                                                \\
\cline{2-5}
                       & Zhu et al.~\cite{2017Machine}                                                     & 20 (10+10)                                                                                                                 & One Rule        & 85.00                                                                                                                                \\
\cline{2-5}
                       & Zhu et al. ~\cite{2020Classifying}                                               & 31 (14+17)                                                                                                                 & RF+eXtreme Gradient Boosting & 92.58                                                                                                                                \\
\cline{2-5}
                       & Ma et al.~\cite{2020DISTINGUISHING }                                              & 36 (MDD)+48 (bipolar disorder patients)                                                                                    & LSTM                                                                                                                  & 96.20                                                                                                                                \\
\cline{2-5}
                       & Wang et al.~\cite{2021Depression}                                                & 96 (79+17)                                                                                                                 & CNN & 90.00                                                                                                                                \\
\hline

\end{tabular}}
           \begin{tablenotes}
			\footnotesize
			\item Least Absolute Shrinkage and Selection Operator(LASSO),
			    standard Least Absolute Shrinkage and Selection Operator (sLASSO),
				group Least Absolute Shrinkage and Selection Operator (gLASSO),
				sparse group Least Absolute Shrinkage and Selection Operator (sgLASSO),
				%Multivariate Pattern Analysis (MVPA),
				%Functional Connectivity Network (FCN),
				%Dynamic Functional Connectivity (DFC),
				%Functional Network Connectivity (FNC),
				%Spatial-temporal CNN (ST-CNN).
		\end{tablenotes}
\end{table*}

%MVPA (Multivariate pattern analysis)

\ifx\allfiles\undefined
% !TEX root = tnnls_relation_gait.tex

% if have a single appendix:
%\appendix[Proof of the Zonklar Equations]
% or
%\appendix  % for no appendix heading
% do not use \section anymore after \appendix, only \section*
% is possibly needed

% use appendices with more than one appendix
% then use \section to start each appendix
% you must declare a \section before using any
% \subsection or using \label (\appendices by itself
% starts a section numbered zero.)
%

%\appendices
%\section{Proof of the First Zonklar Equation}
%Appendix one text goes here.
%
%% you can choose not to have a title for an appendix
%% if you want by leaving the argument blank
%\section{}
%Appendix two text goes here.

% use section* for acknowledgment
% \section*{Acknowledgment}
% The authors would like to thank Prof. Dongbin Zhao for his support to this work.

% Can use something like this to put references on a page
% by themselves when using endfloat and the captionsoff option.
\ifCLASSOPTIONcaptionsoff
  \newpage
\fi

% trigger a \newpage just before the given reference
% number - used to balance the columns on the last page
% adjust value as needed - may need to be readjusted if
% the document is modified later
%\IEEEtriggeratref{8}
% The "triggered" command can be changed if desired:
%\IEEEtriggercmd{\enlargethispage{-5in}}

% references section

% can use a bibliography generated by BibTeX as a .bbl file
% BibTeX documentation can be easily obtained at:
% http://mirror.ctan.org/biblio/bibtex/contrib/dsoc/
% The IEEEtran BibTeX style support page is at:
% http://www.michaelshell.org/tex/ieeetran/bibtex/
\bibliographystyle{IEEEtran}
% argument is your BibTeX string definitions and bibliography database(s)
\bibliography{IEEEabrv,tnnls_relation_gait}
% \bibliography{IEEEabrv,1}
%
% <OR> manually copy in the resultant .bbl file
% set second argument of \begin to the number of references
% (used to reserve space for the reference number labels box)
%\begin{thebibliography}{1}
%\bibitem{IEEEhowto:kopka}
%H.~Kopka and P.~W. Daly, \emph{A Guide to \LaTeX}, 3rd~ed.\hskip 1em plus
%  0.5em minus 0.4em\relax Harlow, England: Addison-Wesley, 1999.
%\end{thebibliography}

% biography section
%
% If you have an EPS/PDF photo (graphicx package needed) extra braces are
% needed around the contents of the optional argument to biography to prevent
% the LaTeX parser from getting confused when it sees the complicated
% \includegraphics command within an optional argument. (You could create
% your own custom macro containing the \includegraphics command to make things
% simpler here.)
%\begin{IEEEbiography}[{\includegraphics[width=1in,height=1.25in,clip,keepaspectratio]{mshell}}]{Michael Shell}
% or if you just want to reserve a space for a photo:

%\begin{IEEEbiography}{Michael Shell}
%Biography text here.
%\end{IEEEbiography}
%
%% if you will not have a photo at all:
%\begin{IEEEbiographynophoto}{John Doe}
%Biography text here.
%\end{IEEEbiographynophoto}

% insert where needed to balance the two columns on the last page with
% biographies
% \newpage

%\begin{IEEEbiographynophoto}{Jane Doe}
%Biography text here.
%\end{IEEEbiographynophoto}


% You can push biographies down or up by placing
% a \vfill before or after them. The appropriate
% use of \vfill depends on what kind of text is
% on the last page and whether or not the columns
% are being equalized.

%\vfill

% Can be used to pull up biographies so that the bottom of the last one
% is flush with the other column.
%\enlargethispage{-5in}

% that's all folks
\end{document} 
\fi 

\subsection{Performance Comparison}
%%%%%%%%%%%%%%%%%%%%%%%%%%%%%%%%%%%%%%%
%Table \ref{tab3} shows a summary of the experimental results based on brain imaging to diagnosing depression.
%As most researchers use their own collected data for the auxiliary diagnosis of depression based on brain images, the work among them is not comparable, but by comparing different methods of the same team, we can draw the following conclusions:
We evaluate the auxiliary depression diagnosis method using fMRI and FNIRS.
We review the effectiveness of the approaches examined in Tabel~\ref{tab3} to provide greater insight into the  brain imaging auxiliary depression diagnosis methods' performance.
The experimental classification accuracy data (\%) are taken directly from the related source articles to conduct a fair comparison.
It should be mentioned that we focus on comparing how different techniques perform on the same dataset.
The following list of observations can be summed up:

(1)Although the general linear model is the mainstream analysis method of fMRI images, based on its good ability to feature extraction, it has also achieved better results than the traditional linear feature extraction methods in fNIRS.

(2)When analyzing fMRI, the CNN based model has a stronger ability to identify the characteristics of depression than the functional connectivity network model.
Deep learning methods have been applied to find spatial temporal relationships between voxels of a stationary state network. Networks like CNN are known to be the best method for finding spatial. 

(3)The lack of training samples is a serious problem when the deep learning method is applied to medical image analysis. Although the GAN model has not achieved good results, such a framework promises wide utility and great potential in neuroimaging biomarker identification.

%including the source and name of the method, the input brain imaging modality, the data set used (- represents that the data used were collected by ourselves), and the evaluation criteria of the experimental results, including Accuracy, FScore Precision, Recall, Sensitivity, Specificity, and Area Under Curve (AUC).

%SVM dominates the machine learning classifiers and usually achieves good results when compared with other classifiers. And deep learning algorithms usually achieve higher recognition accuracy than machine learning classifiers.
