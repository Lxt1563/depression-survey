% !TEX root = tnnls_relation_gait.tex

\ifx\allfiles\undefined
    \input{tnnls_prefix}
\fi
%\section{Depression Recognition}
\section{Depression recognition method based on multi-modal data}

As described in Section 3, most studies have been conducted to detect depression based only on single physiological indicators, such as EEG, EM, speech signals, facial expressions, text, etc.
Although single indicators can identify depression to some extent, they do not give a comprehensive representation.
For example, in case of a depressed person, when a stimulus is presented, speech signals showed a longer response time and lower pronunciation rate~\cite{alghowinem2016multimodal}, while EM  showed decreased eyebrow movement and elevated blink rates ~\cite{sobin1997psychomotor, ellgring2007non, mackintosh1983blink}; thus, it is considerably likely that both modalities are correlated.
It seems obvious that the signal from single modality provided only partial information, while a combination of different modality signals can be used to form a more realistic model for recognizing depression than the former~\cite{gupta2014multimodal}.
So, there is increasing interest in using different modalities to handle information.





In short, a more accurate and robust depression recognition model can be constructed by integrating complementary information from different modalities that reflect different aspects of depression.
In this section, we will generally divide depression detection based on multi-modal data into three categories: depression recognition method based on multiple electrophysiological signals, depression recognition method based on multiple brain imaging, and depression recognition method based on multiple audio and video data.



\subsection{Depression recognition method based on multiple electrophysiological signals }


Various studies have shown that electrophysiological signals such as EEG, ECG, EOG, and EMG can reflect a person's
physiological and emotional state to varying degrees, which is helpful for clinically assisted diagnosis of depression~\cite{sornmo2005bioelectrical}.
Further, electrophysiological signals not only have the benefits of good timing and convenient operation, but also have higher generalizability and stability of the developed depression recognition model~\cite{he2020advances}.
Therefore, with the continuous development and update of machine learning, research into the depression recognition using multiple physiological signals has increasingly attracted attention.
The researches generally adopt feature level fusion or decision level fusion in depression recognition.
%�������ں�

%Due to the limited information recorded by single-modality data, more and more researchers are devoted to the study of multi-modal data fusion.



\subsubsection{Feature level fusion}


\begin{figure}[tbp]
	\centering	
	\label{multi-modal_01}\includegraphics[width=0.9\linewidth]{figures/depression/multi-modal_01.jpg}		
	\caption{
     Flow chart of fusion based on multiple electrophysiological signals (taking two modalities as an example): subfigures a and b represent feature level fusion and decision level fusion, respectively.
	}
	\label{multi-modal_01}
\end{figure}
Feature level fusion, which can be divided into serial and parallel forms, is the fusion of the data that results from feature extraction~\cite{haghighat2016discriminant,yang2003feature}.
Feature level fusion aims to find correlations between data from the extracted features or to evaluate the modalities, which can reduce noise interference and improve model efficiency to some extent.
Taking depression recognition based on multiple electrophysiological signals as an example, the specific process of feature level fusion is shown in Fig~\ref{multi-modal_01}.
In order to identify depressions, Zhu et al.~\cite{zhu2019multimodal} proposed a model based on feature level fusion.
And they validated the presented classification model by collecting two physiological signals, EEG and EM.
The experimental result indicated that the feature fusion method slightly improved the recognition accuracy by 1.88\%, compared with the unimodal classification approach that uses only EEG or EM.
Thus, it was concluded that the feature level fusion methods can improve the mild depression recognition accuracy, demonstrating the complementary nature of the modalities.





\subsubsection{Decision level fusion}


Decision level fusion is the process of taking the information obtained from each modality and making independent decisions, and then fusing the outcomes of these decisions in some ways.
Taking depression recognition based on multiple electrophysiological signals as an example, the specific process of decision level fusion is shown in Fig~\ref{multi-modal_01}.
Among the different fusion strategies, decision level fusion has unique advantages to fuse the output of various classifiers and getting an effective result.
It can also synthesize multi-source information and avoid mutual interference.
However,  the classification results of each modality are usually not completely reliable, and there are misclassifications.
Moreover, the classification results from different modalities of one object may have high conflict, and this is very unfavorable to the fusion result.
To solve these problems, Zhu et al.~\cite{zhu2022content}  proposed a content-based multiple evidence fusion (CBMEF) method, which fused EEG and EM data at decision level.
The method mainly included two modules, the classification performance matrix module and the dual-weight fusion module.
The classification performance matrices of different modalities were estimated by Bayesian rule based on confusion matrix and Mahalanobis distance, and the matrices were used to correct the classification results.
Then the relative conflict degree of each modality was calculated, and different weights were assigned to the above modalities at the decision fusion layer according to this conflict degree.
The idea of introducing the classification performance matrix and the dual-weight model to multimodal biosignals fusion casts a new light on the researches of depression recognition.





\subsection{Others}
%%%%%%%%%%%%%%%%%%%%%%%%%%%%%%%%%%%%%%%%%%%%%%%%%%%%%%%
In addition to the above three types of multimodal fusions, there are a number of other modal fusions that have also attracted extensive attention from researchers.
For example,
Bruder et al.~\cite{bruder2012relationship} combined high temporal resolution EEG signals with MRI data to increase the precision of identifying depression.
Zhang et al.~\cite{zhang2019multimodal} explored from physiological and behavioral perspectives simultaneously and fused pervasive EEG and speech signals to make the detection of depression more objective, effective and convenient.
Jing et al.~\cite{jing2019different}  built a supervised regression model based on personal language and natural gait data to predict depression and anxiety in patients. Their results could be a basis of both applications and future studies on the multi-source data fusion in anxiety and depression recognition.


\ifx\allfiles\undefined
% !TEX root = tnnls_relation_gait.tex

% if have a single appendix:
%\appendix[Proof of the Zonklar Equations]
% or
%\appendix  % for no appendix heading
% do not use \section anymore after \appendix, only \section*
% is possibly needed

% use appendices with more than one appendix
% then use \section to start each appendix
% you must declare a \section before using any
% \subsection or using \label (\appendices by itself
% starts a section numbered zero.)
%

%\appendices
%\section{Proof of the First Zonklar Equation}
%Appendix one text goes here.
%
%% you can choose not to have a title for an appendix
%% if you want by leaving the argument blank
%\section{}
%Appendix two text goes here.

% use section* for acknowledgment
% \section*{Acknowledgment}
% The authors would like to thank Prof. Dongbin Zhao for his support to this work.

% Can use something like this to put references on a page
% by themselves when using endfloat and the captionsoff option.
\ifCLASSOPTIONcaptionsoff
  \newpage
\fi

% trigger a \newpage just before the given reference
% number - used to balance the columns on the last page
% adjust value as needed - may need to be readjusted if
% the document is modified later
%\IEEEtriggeratref{8}
% The "triggered" command can be changed if desired:
%\IEEEtriggercmd{\enlargethispage{-5in}}

% references section

% can use a bibliography generated by BibTeX as a .bbl file
% BibTeX documentation can be easily obtained at:
% http://mirror.ctan.org/biblio/bibtex/contrib/dsoc/
% The IEEEtran BibTeX style support page is at:
% http://www.michaelshell.org/tex/ieeetran/bibtex/
\bibliographystyle{IEEEtran}
% argument is your BibTeX string definitions and bibliography database(s)
\bibliography{IEEEabrv,tnnls_relation_gait}
% \bibliography{IEEEabrv,1}
%
% <OR> manually copy in the resultant .bbl file
% set second argument of \begin to the number of references
% (used to reserve space for the reference number labels box)
%\begin{thebibliography}{1}
%\bibitem{IEEEhowto:kopka}
%H.~Kopka and P.~W. Daly, \emph{A Guide to \LaTeX}, 3rd~ed.\hskip 1em plus
%  0.5em minus 0.4em\relax Harlow, England: Addison-Wesley, 1999.
%\end{thebibliography}

% biography section
%
% If you have an EPS/PDF photo (graphicx package needed) extra braces are
% needed around the contents of the optional argument to biography to prevent
% the LaTeX parser from getting confused when it sees the complicated
% \includegraphics command within an optional argument. (You could create
% your own custom macro containing the \includegraphics command to make things
% simpler here.)
%\begin{IEEEbiography}[{\includegraphics[width=1in,height=1.25in,clip,keepaspectratio]{mshell}}]{Michael Shell}
% or if you just want to reserve a space for a photo:

%\begin{IEEEbiography}{Michael Shell}
%Biography text here.
%\end{IEEEbiography}
%
%% if you will not have a photo at all:
%\begin{IEEEbiographynophoto}{John Doe}
%Biography text here.
%\end{IEEEbiographynophoto}

% insert where needed to balance the two columns on the last page with
% biographies
% \newpage

%\begin{IEEEbiographynophoto}{Jane Doe}
%Biography text here.
%\end{IEEEbiographynophoto}


% You can push biographies down or up by placing
% a \vfill before or after them. The appropriate
% use of \vfill depends on what kind of text is
% on the last page and whether or not the columns
% are being equalized.

%\vfill

% Can be used to pull up biographies so that the bottom of the last one
% is flush with the other column.
%\enlargethispage{-5in}

% that's all folks
\end{document} 
\fi
