% !TEX root = tnnls_depression_survey.tex

\ifx\allfiles\undefined
    \input{tnnls_prefix}
\fi


\subsection{Gait}
%%%%%%%%%%%%%%%%%%%%%%%%%%%%%%%%%%%%%%%%%%%%%%%%%%%%%%%
As a daily movement, human gait occurs parallel with the development of higher brain structures and functions (prefrontal cortex, basal ganglia, and cerebellum). It reflects the integrity of the higher brain systems~\cite{sheridan2007role}; thus, it is a good indicator of mental status.
Furthermore, researchers have also studied the gait characteristics of depression patients through motion analysis~\cite{sloman1987mood,brandler2012depressive,van2005quantitative,michalak2011effects,radovanovic2014gait,naidu2019does}.
Specifically, Lemake et al.~\cite{lemke2000spatiotemporal} calculate the spatiotemporal gait parameters of patients with major depressive disorder and found that the patients show significant reductions in stride length, cycle time, and lower limb support.
Moreover, Michalak~\cite{michalak2009embodiment} found reduced walking velocity, arm swing, vertical body movement, increased body sway, and more slumped posture in patients.

%%%%%%%%%%%%%%%%%%%%%%%%%%%%%%%%%%%%%%%%%%%%%%%%%%%%%%%%%%%%%%%%%%%%%%%%%%%%%%%%%%%%%%%%%%%%%%%%%%%%%
%%%%%%%%%%%%%%%%%%%%%%%%%%%%%%%%%%%%%%%%%%%%%%%%%%%%%%%%%%%%%%%%%%%%%%%%%%%%%%%%%%%%%%%%%%%%%%%%%%%%%

Compared to traditional mental illness detection biometrics, such as facial expression, speech, and physiological parameters, gait is remotely observable, more difficult to imitate, and requires less cooperation from the subject~\cite{xu2020emotion}. These advantages make gait a promising source for an auxiliary depression diagnosis.
Kinect~\cite{gholami2016microsoft}, a device with depth and infrared sensors, can record data from many modalities, such as skeletal and RGB images while walking.
It enables the acquisition of gait information and the development of gait auxiliary depression diagnosis.
There are two basic approaches in related work: traditional machine learning and deep learning.
Notably, these methods concentrate on two tasks: regression of scale scores and depression classification.
This section mainly presents work on the classification task, as the number of regression tasks involved is small.
%%%%%%%%%%%%%%%%%%%%%%%%%%%%%%%%%%%%%%%%%%%%%%%%%%%%%%%%%%%%%%%%%%%%%%%%%%%%%%%%%%%%%%%%%%%%%%%%%%%%%
%%%%%%%%%%%%%%%%%%%%%%%%%%%%%%%%%%%%%%%%%%%%%%%%%%%%%%%%%%%%%%%%%%%%%%%%%%%%%%%%%%%%%%%%%%%%%%%%%%%%%


%%%%%%%%%%%%%%%%%%%%%%%%%%%%%%%%%%%%%%%%%%%%%%%%%%%%%%%%%%%%%%%%%%%%%%%%%%%%%%%%%%%%%%%%%%%%%%%%%%%%%%%%%%%%%%
%%%%%%%%%%%%%%%%%%%%%%%%%%%%%%%%%%%%%%%%%%%%%%%%%%%%%%%%%%%%%%%%%%%%%%%%%%%%%%%%%%%%%%%%%%%%%%%%%%%%%%%%%%%%%%
We first explain the two gait representations that are the basis of the study: skeletal and silhouette, to better comprehend gait auxiliary depression diagnosis before introducing the two types of approaches discussed above.
The skeleton sequence refers to the 3D coordinate data that can characterize the human body behaviour, which is abstracted from the human body structural model of each critical joint of the human body~\cite{nordin2016survey}.
The skeleton-based representation has strong stability, is not easily influenced by the observational perspective, and can accurately depict the variations in human action~\cite{wang2016gait}.
The extraction of the silhouette is usually obtained by background subtraction and followed by binarization. Gait silhouette removes all the colour and texture information and makes recognition focus on gait, not clothing or other non-gait features.
The gait energy image~\cite{han2005individual}, which is created by averaging the binary silhouette of the human body, has become a popular silhouette feature due to its straightforward processing and potent anti-noise properties~\cite{yang2008gait}.
%%%%%%%%%%%%%%%%%%%%%%%%%%%%%%%%%%%%%%%%%%%%%%%%%%%%%%%%%%%%%%%%%%%%%%%%%%%%%%%%%%%%%%%%%%%%%%%%%%%%%%%%%%%%%%%
%%%%%%%%%%%%%%%%%%%%%%%%%%%%%%%%%%%%%%%%%%%%%%%%%%%%%%%%%%%%%%%%%%%%%%%%%%%%%%%%%%%%%%%%%%%%%%%%%%%%%%%%%%%%%%%

\begin{figure}[tbp]
\centering
\includegraphics[width=0.9\linewidth]{figures/depression/kinect.jpg}
\caption{Video based gait acquisition system.
}
\label{kinect}
\end{figure}


\subsubsection{Gait auxiliary depression diagnosis based on traditional machine learning}
%Gait auxiliary depression diagnosis is still in its infancy, with most approaches focusing on traditional machine learning.
In traditional research of gait auxiliary depression diagnosis, hand-crafted features are regularly used together with machine learning classifiers. The gait features are generally divided into two main categories: time-frequency and spatial-temporal features.



%%%%%%%%%%%%%%%%%%%%%%%%%%%%%%%%%%%%%%%%%%%%%%%%%%%%%%%%%%%%%%%%%%%%%%%%%%%%%%%%%%%%%%%%%%%%%%%%%%%%%%%%%%%%%%%%%%%
%%%%%%%%%%%%%%%%%%%%%%%%%%%%%%%%%%%%%%%%%%%%%%%%%%%%%%%%%%%%%%%%%%%%%%%%%%%%%%%%%%%%%%%%%%%%%%%%%%%%%%%%%%%%%%%%%%%

The time-frequency analysis is a powerful tool in signal processing for investigating a signal simultaneously in the time and frequency domains.
Since the gait data is simply samplings of time series, the time-frequency analysis time-frequency analysis is an effective method for for identifying features describing gait.
Specifically, there are four commonly used gait time-domain features: mean, standard deviation, skewness, and kurtosis of the original data.
The mean is a measure of the central tendency of the random variable characterized by that distribution.
Standard deviation measures the amount of dispersion of a dataset.
Kurtosis measures outliers present in the probability distribution.
In the meantime, several features in the frequency domain, such as amplitudes of all component frequencies, phases, etc., can be obtained with the help of the Fast Fourier transform. It can also be used to keep the critical components while reducing the dimensionality of the gaits data. These temporal and frequency data have a strong correlation with depression, and some techniques, such as SVM, etc., can use them as classification features for auxiliary depression diagnosis.
Furthermore, Wang et al.~\cite{wang2021detecting} validate the effectiveness of both times- and frequency-domain features in recognizing depression, which is also consistent with previous studies~\cite{sejdic2013comprehensive,zhang2016emotion}.
However, significant time- and frequency-domain principal components can only provide them with mathematical relationships
between gait patterns and time- and frequency-domain features. They need to be more intuitive for interpreting gait patterns.


Differently, while high-level spatial-temporal features may provide an intuitive understanding
of individual gait patterns, they contribute less to auxiliary depression diagnosis than low-level time-and frequency-domain features~\cite{wang2021detecting}.
The spatiotemporal gait features, which comprise the distance between joints, the angle created by the joints, and the distance between the joint point and the bone, represent changes in the distance angle and other properties of the joints with time.
Additionally, there are noticeable differences between the depressed and non-depressed groups in several spatiotemporal, kinematic, and postural gait characteristics, including walking speed, stride length, head movement, vertical head posture, arm swing, and body sway.
Applying these features to classifiers with various machine learning techniques, Fang et al.~\cite{fang2019depression} investigate whether natural gait analysis might be a practical and reliable approach to support auxiliary depression diagnosis.
The results suggest that spatial features are beneficial in assessing depression.
Furthermore, reasonable accuracy can already be achieved by using parameters from the upper body alone, indicating that upper body postures and movements can effectively contribute to depression analysis.

%%%%%%%%%%%%%%%%%%%%%%%%%%%%%%%%%%%%%%%%%%%%%%%%%%%%%%%%%%%%%%%%%%%%%%%%%%%%%%%%%%%%%%%%%%%%%%%%%%%%%%%%%%%%%%%%%%%%
%%%%%%%%%%%%%%%%%%%%%%%%%%%%%%%%%%%%%%%%%%%%%%%%%%%%%%%%%%%%%%%%%%%%%%%%%%%%%%%%%%%%%%%%%%%%%%%%%%%%%%%%%%%%%%%%%%%%



%Traditional classification or regression algorithms are employed after feature extraction such as SVM, LR, RF, Decision Tree, GMM, KNN, etc.
%Li \cite{li2021simple} used the Kinect V2 device to record kinematic skeleton data of the participants' 25 body joints, the presented spatial features and low-level features is directly extracted from the record original Kinect-3D coordinates.
%The scored-depressed and non-depressed individuals can be well classified by computational models which were import processed data directly. The proposed experiment demonstrated four strong machine learning tools: SVM, LR, RF and Gradient Boosting (GB).



\subsubsection{Gait auxiliary depression diagnosis based on deep learning}
Depending on the input data, deep learning methods for gait auxiliary depression diagnosis can group into two categories. One is a deep end-to-end model with a skeleton sequence as input, and the other is the deep hybrid model with a skeleton sequence and silhouette as input.

With deep end-to-end models, the skeleton sequence is directly pushed into the neural network, enabling the model to learn high-level characteristics and produce classification output autonomously.
LSTM and Temporal Convolutional Networks (TCN) are today's most widely used depth models.
In the LSTM, the recurrent structure inside the network is constructed by taking the last time's output as the current time's input, which is an efficient way to handle sequential data.
The skeleton sequence is a natural time series of joint coordinates that can be thought of as a sequential vector and is thus ideally suited for LSTM processing.
The TCN learns time-dimensional nodal change features by performing temporal convolution operations on time-dimensional input using conventional convolutional layers.
Yang et al.~\cite{yang2022data} use LSTM and TCN for an auxiliary depression diagnosis. While they produce good results, LSTM and TCN, which focus solely on inter-frame information, lack spatial information construction.
In order to get around this limitation and effectively gather spatiotemporal information from 3D skeletons, Graph Convolutional Networks (GCN) can be used for skeleton data processing in the future. GCN is the most realistic representation of human skeletal joints since it is based on the topological map approach.


The deep hybrid model individually learns the skeleton sequence and silhouette features and combines their predictions at the decision level.
The skeleton can help to locate the spatial position of joints of the human body, and the silhouette can describe detailed information about the human body shape.
The deep hybrid model can take full advantage of the complementing nature of skeleton and silhouette modes, which helps to increase the model's performance and generalization.
Shao et al.~\cite{shao2021multi} propose a hybrid model-based gait auxiliary depression diagnosis method that combines skeleton modality and silhouette modality. Firstly, they propose a skeleton feature set to describe depression and train an LSTM model to conduct sequence strategy. Secondly, they generate GEI as silhouette features from RGB videos and design two CNN models with a new loss function to extract silhouette features from front and side perspectives.
Then, they construct a fusion model consisting of fusing silhouettes from the front and side views at the feature level and the classification results of different modalities at the decision level.
The experimental results imply that the proposed method leveraging the complementarity gained a higher accuracy. However, the feature extraction of the skeleton alone and the generation of GEI with silhouettes suffer from variable degrees of information loss. Instead, employing a deep end-to-end architecture to push the original signal into the model for learning may produce unpredictable results.

\subsubsection{Performance Comparison}
%To shed more light on the performance of gait auxiliary depression diagnosis methods, we summarize the performance of the methods tested in Tabel~\ref{tab_gait}.
%To perform a fair comparison, the experimental classification accuracy results (\%) are directly derived from the corresponding original papers.
%It should be noted that we concentrate on contrasting the performance of various approaches on the same data set.
%Several observations can be summarized as follows:

We provide a performance summary of the methods evaluated in Tabel~\ref{tab_gait} to provide additional insight into the effectiveness of gait auxiliary depression diagnosis approaches.
The experimental classification accuracy (\%) is taken directly from the related source articles to conduct a fair comparison.
We should mention that we focus on comparing how different techniques perform on the same dataset.
The following list of observations can be summed up:


(1) 
For traditional gait auxiliary depression diagnosis methods, scholars have been exploring novel gait features and using traditional machine learning models for classification with increasing accuracy.
Specifically, first, time- and frequency-domain features are more efficient in constructing computational models to identify depression than spatiotemporal features~\cite{wang2021detecting}. It is noteworthy that the model built with time-domain features has the same performance as the model built by combining both spatiotemporal and time-domain features, indicating that spatiotemporal features contribute little to identifying depression. It is also reflected in the fact that the model comprised of time- and frequency-domain features has the same performance as the model consisting of all features. Second, different classifiers display varying performances when given the same feature vector as input.


(2) For deep learning-based gait auxiliary depression diagnosis methods, the multi-modal models are more accurate than the single-modal models. Researchers found that the skeleton can help to locate the spatial position of joints of the human body, and the silhouette can describe the detailed information of the human body shape so that the constructed 3D model is more vivid.
Shao et al.~\cite{shao2021multi} obtain better classification performance using skeleton and silhouette fusion, which agrees that the skeleton and silhouette information are related and complementary.


(3)
The gait auxiliary depression diagnosis method based on deep learning outperforms traditional machine learning for the same dataset~\cite{9187648,fang2019depression,yang2022data}.
In previous studies, scholars have been committed to exploring new gait features and using traditional machine learning models for classification, and the accuracy has been continuously improved.
However, traditional machine learning technologies need to be improved in processing raw data in their raw form. They could not express complex functions, making solving more complex natural signal processing problems difficult. Deep learning allows computational models composed of multiple processing layers to learn representations of data with multiple levels of abstraction and demonstrate a powerful ability to learn the essential characteristics of a data set from a small number of samples.




\begin{table}[!htbp]
\caption{Experimrntal results based on gait.}
\label{tab_gait}
\renewcommand\arraystretch{1.2}
\resizebox{0.9999\linewidth}{!}{%
\begin{tabular}{@{}c|c|c|c@{}}
\toprule
\textbf{Paper} & \textbf{Dataset(D+C)}& \textbf{Methods} & \textbf{Methics(Accuracy/\%)} \\ \midrule
Wang~\cite{9187648}                  & 95(43+52)                   & FC  & 93.75  \\ \cline{1-4}
Yuan~\cite{yuan2018depression}  & 101(54+47) & SVM  & 91.09 \\ \cline{1-4}
Wang~\cite{wang2021detecting} & 247(126+121) & SVM  & AUC=93.00 \\ \cline{1-4}
Zhao~\cite{zhao2019see}    & 179    & SVR   & 64.00  \\ \cline{1-4}
\multirow{4}{*}{Li~\cite{li2021simple}}   & \multirow{4}{*}{179(85+85)} & SVM  & 53.85   \\
&  & RF  & 61.54    \\
&  & LR  & 73.08   \\
&  & GB  & 76.92    \\ \cline{1-4}
\multirow{3}{*}{Lu~\cite{lu2022postgraduate}}   & \multirow{3}{*}{95(43+52)}  & KNN & 80.34  \\
&  & SVM & 91.21  \\
&   & RF  & 85.71  \\ \cline{1-4}
\multirow{5}{*}{Fang~\cite{fang2019depression}} & \multirow{5}{*}{95(43+52)}  & SVM  & 90.53   \\
&  & RF  & 91.58   \\
& & KNN  & 87.37                         \\
&  & LR    & 88.42                         \\
&  & LDA  & 88.42    \\ \cline{1-4}
Shao~\cite{shao2021multi}  & 200(86+114)  & LSTM+CNN  & 85.45 \\ \cline{1-4}
\multirow{2}{*}{Yang~\cite{yang2022data}} & \multirow{2}{*}{95(43+52)}  & LSTM   & 90.35 \\
&     & TCN    & 92.15     \\ \bottomrule
\end{tabular} }
\end{table}



\ifx\allfiles\undefined
% !TEX root = tnnls_relation_gait.tex

% if have a single appendix:
%\appendix[Proof of the Zonklar Equations]
% or
%\appendix  % for no appendix heading
% do not use \section anymore after \appendix, only \section*
% is possibly needed

% use appendices with more than one appendix
% then use \section to start each appendix
% you must declare a \section before using any
% \subsection or using \label (\appendices by itself
% starts a section numbered zero.)
%

%\appendices
%\section{Proof of the First Zonklar Equation}
%Appendix one text goes here.
%
%% you can choose not to have a title for an appendix
%% if you want by leaving the argument blank
%\section{}
%Appendix two text goes here.

% use section* for acknowledgment
% \section*{Acknowledgment}
% The authors would like to thank Prof. Dongbin Zhao for his support to this work.

% Can use something like this to put references on a page
% by themselves when using endfloat and the captionsoff option.
\ifCLASSOPTIONcaptionsoff
  \newpage
\fi

% trigger a \newpage just before the given reference
% number - used to balance the columns on the last page
% adjust value as needed - may need to be readjusted if
% the document is modified later
%\IEEEtriggeratref{8}
% The "triggered" command can be changed if desired:
%\IEEEtriggercmd{\enlargethispage{-5in}}

% references section

% can use a bibliography generated by BibTeX as a .bbl file
% BibTeX documentation can be easily obtained at:
% http://mirror.ctan.org/biblio/bibtex/contrib/dsoc/
% The IEEEtran BibTeX style support page is at:
% http://www.michaelshell.org/tex/ieeetran/bibtex/
\bibliographystyle{IEEEtran}
% argument is your BibTeX string definitions and bibliography database(s)
\bibliography{IEEEabrv,tnnls_relation_gait}
% \bibliography{IEEEabrv,1}
%
% <OR> manually copy in the resultant .bbl file
% set second argument of \begin to the number of references
% (used to reserve space for the reference number labels box)
%\begin{thebibliography}{1}
%\bibitem{IEEEhowto:kopka}
%H.~Kopka and P.~W. Daly, \emph{A Guide to \LaTeX}, 3rd~ed.\hskip 1em plus
%  0.5em minus 0.4em\relax Harlow, England: Addison-Wesley, 1999.
%\end{thebibliography}

% biography section
%
% If you have an EPS/PDF photo (graphicx package needed) extra braces are
% needed around the contents of the optional argument to biography to prevent
% the LaTeX parser from getting confused when it sees the complicated
% \includegraphics command within an optional argument. (You could create
% your own custom macro containing the \includegraphics command to make things
% simpler here.)
%\begin{IEEEbiography}[{\includegraphics[width=1in,height=1.25in,clip,keepaspectratio]{mshell}}]{Michael Shell}
% or if you just want to reserve a space for a photo:

%\begin{IEEEbiography}{Michael Shell}
%Biography text here.
%\end{IEEEbiography}
%
%% if you will not have a photo at all:
%\begin{IEEEbiographynophoto}{John Doe}
%Biography text here.
%\end{IEEEbiographynophoto}

% insert where needed to balance the two columns on the last page with
% biographies
% \newpage

%\begin{IEEEbiographynophoto}{Jane Doe}
%Biography text here.
%\end{IEEEbiographynophoto}


% You can push biographies down or up by placing
% a \vfill before or after them. The appropriate
% use of \vfill depends on what kind of text is
% on the last page and whether or not the columns
% are being equalized.

%\vfill

% Can be used to pull up biographies so that the bottom of the last one
% is flush with the other column.
%\enlargethispage{-5in}

% that's all folks
\end{document} 
\fi
