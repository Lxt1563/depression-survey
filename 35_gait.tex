% !TEX root = tnnls_depression_survey.tex

\ifx\allfiles\undefined
    \input{tnnls_prefix}
\fi

%\section{Depression Recognition}
%\section{Depression recognition method based on audiovisual data}
\subsection{Gait}
%%%%%%%%%%%%%%%%%%%%%%%%%%%%%%%%%%%%%%%%%%%%%%%%%%%%%%%
Human gait, as a daily movement, occurs in parallel with the development of higher brain structures and functions (prefrontal cortex, basal ganglia, and cerebellum) and reflects the integrity of the higher brain systems\cite{sheridan2007role}; thus, it is a good indicator of mental status.
Compared to traditional mental illness detection biometrics, such as facial expression, speech, and physiological parameters, gait is remotely observable, more difficult to imitate, and requires less cooperation from the subject\cite{xu2020emotion}. These advantages make gait a promising source for depression recognition.

Furthermore, researchers have also studied the gait characteristics of depression patients through motion analysis~\cite{sloman1987mood,brandler2012depressive,van2005quantitative,michalak2011effects,radovanovic2014gait,naidu2019does}.
Specifically, Lemake et al.~\cite{lemke2000spatiotemporal} calculated the spatiotemporal gait parameters of patients with major depressive disorder and found that the patients show significant reductions in stride length, cycle time, and lower limb support.
Moreover, by using the motion capture system with video cameras, Michalak\cite{michalak2009embodiment} found reduced walking velocity, arm swing, vertical body movement, increased body sway, and more slumped posture in patients.
Thanks to the development of consumer-grade depth cameras~\cite{karg2010recognition,venture2014recognizing} (e.g., Kinect\cite{gholami2016microsoft}, the data collection
scheme is shown in Fig~\ref{kinect}), human body dynamics can be traced accurately in a three-dimensional manner without the assistance of the attached markers or the requirement of a specifically designed environment.
The recorded depth and skeleton data have been widely and successfully used in depression recognition.
Therefore, in the current studies of depression recognition, the main work is to learn gait features from the skeleton coordinates recorded by Kinect related with depression and explore feature sets for better performance.
In the meantime, traditional machine learning algorithms are employed to classify such as SVM, KNN, RF, etc.  However, deep learning methods have made breakthroughs in Computer Vision (CV) research. Therefore, many studies are shifting from the traditional hand-crafted gait features to the framework based on deep learning for gait depression recognition.



The two gait representations - the skeleton and the silhouette - that serve as the foundation of the study must first be explained before introducing the two categories of approaches indicated above.
However, the skeleton-based structural model is currently more widely used in gait depression recognition.
It can estimate the spatial location information of human joint points based on the human body structure~\cite{nordin2016survey}. Moreover, it is insensitive to changes in clothes, changes in carrying things, etc \cite{wang2016gait}.
The researchers proposed skeleton features describing the walking process for the skeleton modality, including spatial-temporal, time-domain, and frequency-domain features~\cite{verlekar2019gait}.
On the other hand, gait energy image (GEI) is a widely used silhouette feature~\cite{han2005individual}.
Benefiting from the cycle length-based time and space normalization, GEI has strong anti-noise capabilities while reducing calculation and memory without loss of performance~\cite{yang2008gait}.

\begin{figure}[tbp]
\centering
\includegraphics[width=0.9\linewidth]{figures/depression/kinect.jpg}
\caption{Video based gait acquisition system.
}
\label{kinect}
\end{figure}


\subsubsection{Gait auxiliary depression diagnosis based on traditional machine learning}
Gait depression recognition is still in its infancy, with most approaches focusing on traditional machine learning.
As shown in Fig~\ref{gait_traditional}, researchers extracted the spatial-temporal, time-domain, and frequency-domain features from the skeleton and input them into machine learning classifiers (such as SVM) for depression recognition.

\begin{figure}[tbp]
\centering
\includegraphics[width=0.9\linewidth]{figures/depression/gait_traditional.jpg}
\caption{Flow chart of gait depression recognition based on traditional machine learning.
}
\label{gait_traditional}
\end{figure}

In traditional research of gait depression recognition, hand-crafted features are regularly used together with statistical features. The commonly used gait features are as follows:
(1) Time-Domain Features: Gait time-Domain features include mean, standard deviation, skewness, and kurtosis of the original data.
Specifically, the mean is a measure of the central tendency of the random variable characterized by
that distribution.
Standard deviation measures the amount of dispersion of a dataset.
Kurtosis measures outliers present in the probability distribution.
(2)Frequency-Domain Features:
Continuous motion trajectories can be considered as signals.
By performing frequency analysis method on them, the frequency features of gait can be obtained.
The discrete Fourier transform, PSD~\cite{tuzel2007human}, and  Hilbert-Huang
transform are commonly used techniques for translating time domain data to frequency domain signals.
(3)spatial-temporal Features: Spatiotemporal gait parameters, including gait velocity, stride length, cadence, stance phase, swing phase etc., provide basic data characterizing a subjects gait.

Previous studies showed that gait-related features could predict depression accurately, and Kinect provided objective and easily accessible data.
Wang et al.\cite{wang2021detecting} used skeleton data estimated by Kinect devices to
predict depression risk in the dataset of 43 scored-depressed and 52 non-depressed individuals. They combined the time domain information, frequency domain information, and spatial geometric features of gait information. The experimental results show that spatial features help a lot in evaluating depression.


Traditional classification or regression algorithms are employed after feature extraction such as SVM, LR, RF, Decision Tree, GMM, KNN, etc.
Li \cite{li2021simple} used the Kinect V2 device to record kinematic skeleton data of the participants' 25 body joints, the presented spatial features and low-level features is directly extracted from the record original Kinect-3D coordinates.
The scored-depressed and non-depressed individuals can be well classified by computational models which were import processed data directly. The proposed experiment demonstrated four strong machine learning tools: SVM, LR, RF and Gradient Boosting (GB).



\subsubsection{Gait auxiliary depression diagnosis based
on deep learning}
Due to the successful application in CV, deep learning is introduced to gait depression recognition.
Unlike traditional methods, once the model and parameters are established, no further human involvement is required.
The essence of deep learning is to learn high-level features automatically by building more hidden layer models to improve the accuracy of classification or score prediction.

\begin{figure}[tbp]
\centering
\includegraphics[width=0.98\linewidth]{figures/depression/gait_deeplearning.jpg}
\caption{Two different approaches for using deep learning models in gait depression recognition: (a) Building a structure combined skeleton feature set and silhouette features with deep learning method; (b) Building an end-to-end deep architecture.
}
\label{gait_deeplearning}
\end{figure}

According to Fig~\ref{gait_deeplearning}, there are two ways to employ deep learning in this field: (1) Build a structure that combines skeleton feature set and silhouette features with the deep learning method.
The key to the skeleton study is the data quality of joint coordinates, which is easily affected by distance, light, and other environments. The skeleton estimation is not accurate in complex scenes.
The difficulty of  silhouettes study is the cross-view recognition, and the recognition results vary greatly under different views. Multimodal fusion can improve the recognition accuracy and generalization performance of the model by exploiting the correlation between each modality. (2) Build an end-to-end deep architecture and then push skeleton sequence into deep architecture to let model learn high-level features by itself.


By merging skeleton characteristics with silhouette features(GEI) for depth feature identification, significant performance is achieved in the detection of gait depressions, and the generalization performance of the model can be enhanced.
Shao \cite{shao2021multi}  proposed a multi-modal depression recognition method by combining skeleton data and silhouette data of gait.
Firstly, for skeleton data, the skeleton feature set for depression recognition was proposed, which
included spatiotemporal features and kinematics features. For silhouette data, a CNN model with a new loss function was designed to extract silhouette features. Finally, they merged the silhouette features of the front and side views; then in the decision level, they fused the classification results of skeleton data and silhouette data.


Compared with methods of performing feature extraction and classification separately, an end to end deep architecture pushes raw signal into its model to learn and give results.
End to end deep architectures have advantages like that it does not require scholars to have a priori
knowledge, deep networks can learn better features and give better classification result.
However, there are a few issues which limit these architectures, such as large-scale data supporting, over-fitting easily, and poor interpretability.
Specifically, there is often a lack of relevant experimental data or difficulty in collecting large amounts of data in the actual research process.
Data size is gradually emerging as one of the constraints restricting further research on automatic depression recognition because models perform poorly with less data.
Data augmentation is a widely used technique in image processing that, to a certain extent, can address the over-fitting issue brought on by insufficient data ~\cite{zhong2020random,cubuk2018autoaugment,cubuk2019autoaugment}.
It includes a set of techniques that can augment the size and quality of training datasets, and it has been proved to be adequate for improving the performance of the deep learning models~\cite{hussain2017differential}.
Yang \cite{yang2022data} proposed a skeleton data augmentation method based on Kinect V2 to evaluate the risk of depression. First, Yang proposed five techniques to enhance skeleton data and applied it to depression and emotion datasets. Then, according to mutual information and classification accuracy, the augmentation methods are divided into two categories (non-noise augmentation and noise augmentation). The results showed that this data augmentation method greatly improves the recognition effect of the classifier.



\subsubsection{Performance Comparison}
To shed more light on the performance of gait depression recognition methods, we summarize the performance of the methods tested in Tabel~\ref{tab_gait}.
To perform a fair comparison, the experimental classification accuracy results (\%) are directly derived from the corresponding original papers.
It should be noted that we concentrate on contrasting the performance of various approaches on the same data set.
Several observations can be summarized as follows:


(1) For traditional gait depression recognition methods, scholars have been working on exploring novel gait features and using traditional machine learning models for classification with increasing accuracy.
Specifically, first, time- and frequency-domain features are more efficient in constructing computational models to identify depression than spatiotemporal features~\cite{wang2021detecting}. It is worth noting that the models built of time- domain features and spatiotemporal and time- domain features have the same performances, which suggests that spatiotemporal features had very few contributions to recognize depression. It is also reflected in the fact that the model comprised of time- and frequency-domain features has the same performance as the model consisted of all features. Second, different classifiers display varying performance when given the same feature vector as input.


(2) For deep learning-based gait depression recognition methods, the multi-modal models are more accurate than the single-modal models~\cite{shao2021multi}. Researchers found that the skeleton can help to locate the spatial position of joints of the human body, and the silhouette can describe the detailed information of the human body shape so that the constructed 3D model is more vivid.
Shao obtained better classification performance using skeleton and silhouette fusion, which agrees that the skeleton and silhouette information are related and complementary.


(3)
The gait depression recognition method based on deep learning outperforms traditional machine learning for the same dataset~\cite{9187648,fang2019depression,yang2022data}.
In previous studies, scholars have been committed to exploring new gait features and using traditional machine learning models for classification, and the accuracy has been continuously improved.
However, traditional machine learning technologies were limited in their ability to process natural data in their raw form, and lack the ability to express complex functions, making it difficult to solve more complex natural signal processing problems. Deep learning allows computational models that are composed of multiple processing layers to learn representations of data with multiple levels of abstraction, and demonstrate a powerful ability to learn the essential characteristics of a data set from a small number of samples.


\begin{table*}[!htbp]
\caption{Performance of gait auxiliary depression diagnosis based on machine learning.}
\label{tab_gait}
\renewcommand\arraystretch{1.5}
\resizebox{0.9999\linewidth}{!}{%
\begin{threeparttable}
\begin{tabular}{c|cccccc}
\toprule
\textbf{Methods}& \textbf{Paper}& \textbf{Dataset}& \textbf{Feature}& \textbf{Classification}& \textbf{Metrics}           & \multicolumn{1}{c}{\textbf{Value}} \\ \midrule
& & & & SVM & & 53.85 \\
& & & & RF & & 61.54\\
& & & & LR & & 73.08 \\
& \multirow{-4}{*}{Li\cite{li2021simple}}& \multirow{-4}{*}{85SD+85C}& \multirow{-4}{*}{skeleton sequential data} & GB & \multirow{-4}{*}{Accuracy} & 76.92 \\ \cline{2-7}
& & & Velocity & & & 88.84+7.13 \\
& & & Angle Velocity & & & 83.93+7.33\\
& & & TF(Pseudo-Velocity Model) & & & 91.96+5.33 \\
& & & SG & & & 70.98+11.85  \\
Gait depression recognition & \multirow{-5}{*}{Wang\cite{9187648}} & \multirow{-5}{*}{43SD+52C}   & TF+SG                                      & \multirow{-5}{*}{FC+Softmax} & \multirow{-5}{*}{Accuracy} & 93.75+2.98 \\ \cline{2-7}
based on traditional & & & & KNN & Accuracy & 80.34 \\
machine learning& & & & SVM & & 91.21  \\
& \multirow{-3}{*}{Lu\cite{lu2022postgraduate}}& \multirow{-3}{*}{43SD+52C} &\multirow{-3}{*}{the joint energy feature} & RF & & 85.71\\ \cline{2-7}
& Yuan\cite{yuan2018depression} & 54SD+47C& FF & SVM & Accuracy & 91.09  \\ \cline{2-7}
& & & walking speed & SVM & & 90.53\\
& & & arm swing,stride length,& RF & & 91.58 \\
& & & vertical head movement, & KNN & & 87.37 \\
& & & body sway,step width, & LR & & 88.42 \\
& \multirow{-5}{*}{Fang\cite{fang2019depression}}& \multirow{-5}{*}{43SD+52C} & joint rom,stance duration & LDA                                       & \multirow{-5}{*}{Accuracy} & 88.42 \\ \cline{2-7}
& & & SF & & & 0.58\\
& & & TF & & & 0.83 \\
& & & FF & & & 0.87 \\
& & & SF+TF & & & 0.83\\
& & & SF+FF & & & 0.83   \\
& & & TF+FF& & & 0.93  \\
& \multirow{-7}{*}{Wang\cite{wang2021detecting}}&\multirow{-7}{*}{126SD+121C}&ALL&\multirow{-7}{*}{SVM}&\multirow{-7}{*}{AUC} & 0.93 \\ \cline{2-7}
& Zhao\cite{zhao2019see} & 179 & FFT & SVR & Accuracy & 64  \\ \midrule
Gait depression recognition &  & & Skeleton features set,GEI  &  &  & 85.45\\
recognition based on & &  & GEI & &  & 66.67   \\
deep learning& \multirow{-3}{*}{Shao\cite{shao2021multi}}& \multirow{-3}{*}{86SD+114C}  & skeleton feature set                       & \multirow{-3}{*}{LSTM+CNN} & \multirow{-3}{*}{Accuracy} & 80.28 \\ \cline{2-7}
& Yang\cite{yang2022data} & 43SD+52C & skeleton data & LSTM & Accuracy  & 92.15  \\ \bottomrule
\end{tabular}
      \begin{tablenotes}
		\item Score-depressed: SD; Time-Domain Features: TF; Frequency-Domain Features: FF; Spatial geometric feature: SG; spatial-temporal Features: SF
     \end{tablenotes}
\end{threeparttable}}
\end{table*}



\ifx\allfiles\undefined
% !TEX root = tnnls_relation_gait.tex

% if have a single appendix:
%\appendix[Proof of the Zonklar Equations]
% or
%\appendix  % for no appendix heading
% do not use \section anymore after \appendix, only \section*
% is possibly needed

% use appendices with more than one appendix
% then use \section to start each appendix
% you must declare a \section before using any
% \subsection or using \label (\appendices by itself
% starts a section numbered zero.)
%

%\appendices
%\section{Proof of the First Zonklar Equation}
%Appendix one text goes here.
%
%% you can choose not to have a title for an appendix
%% if you want by leaving the argument blank
%\section{}
%Appendix two text goes here.

% use section* for acknowledgment
% \section*{Acknowledgment}
% The authors would like to thank Prof. Dongbin Zhao for his support to this work.

% Can use something like this to put references on a page
% by themselves when using endfloat and the captionsoff option.
\ifCLASSOPTIONcaptionsoff
  \newpage
\fi

% trigger a \newpage just before the given reference
% number - used to balance the columns on the last page
% adjust value as needed - may need to be readjusted if
% the document is modified later
%\IEEEtriggeratref{8}
% The "triggered" command can be changed if desired:
%\IEEEtriggercmd{\enlargethispage{-5in}}

% references section

% can use a bibliography generated by BibTeX as a .bbl file
% BibTeX documentation can be easily obtained at:
% http://mirror.ctan.org/biblio/bibtex/contrib/dsoc/
% The IEEEtran BibTeX style support page is at:
% http://www.michaelshell.org/tex/ieeetran/bibtex/
\bibliographystyle{IEEEtran}
% argument is your BibTeX string definitions and bibliography database(s)
\bibliography{IEEEabrv,tnnls_relation_gait}
% \bibliography{IEEEabrv,1}
%
% <OR> manually copy in the resultant .bbl file
% set second argument of \begin to the number of references
% (used to reserve space for the reference number labels box)
%\begin{thebibliography}{1}
%\bibitem{IEEEhowto:kopka}
%H.~Kopka and P.~W. Daly, \emph{A Guide to \LaTeX}, 3rd~ed.\hskip 1em plus
%  0.5em minus 0.4em\relax Harlow, England: Addison-Wesley, 1999.
%\end{thebibliography}

% biography section
%
% If you have an EPS/PDF photo (graphicx package needed) extra braces are
% needed around the contents of the optional argument to biography to prevent
% the LaTeX parser from getting confused when it sees the complicated
% \includegraphics command within an optional argument. (You could create
% your own custom macro containing the \includegraphics command to make things
% simpler here.)
%\begin{IEEEbiography}[{\includegraphics[width=1in,height=1.25in,clip,keepaspectratio]{mshell}}]{Michael Shell}
% or if you just want to reserve a space for a photo:

%\begin{IEEEbiography}{Michael Shell}
%Biography text here.
%\end{IEEEbiography}
%
%% if you will not have a photo at all:
%\begin{IEEEbiographynophoto}{John Doe}
%Biography text here.
%\end{IEEEbiographynophoto}

% insert where needed to balance the two columns on the last page with
% biographies
% \newpage

%\begin{IEEEbiographynophoto}{Jane Doe}
%Biography text here.
%\end{IEEEbiographynophoto}


% You can push biographies down or up by placing
% a \vfill before or after them. The appropriate
% use of \vfill depends on what kind of text is
% on the last page and whether or not the columns
% are being equalized.

%\vfill

% Can be used to pull up biographies so that the bottom of the last one
% is flush with the other column.
%\enlargethispage{-5in}

% that's all folks
\end{document} 
\fi
