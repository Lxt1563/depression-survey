% !TEX root = tnnls_depression_survey.tex

\ifx\allfiles\undefined
    \input{tnnls_prefix}
\fi
%\section{Depression Recognition}
%\section{Conclusion and Future Work}

\section{Discussion}

After a review of the currently existing methods for multi-modal data-based auxiliary depression diagnosis, the following future opportunities have been outlined from four perspectives: task, dataset, model, and application.
\subsection{Task}

%We first discuss several potential directions for innovating deep long-tailed learning methods.
%
% there are several new task settings
%of long-tailed learning waiting to be resolved.
There are several new character settings that need to be resolved in order to completely and deeply explore depression.

\subsubsection{Sub-types of depression}
The existing studies typically distinguish depressed and normal individuals or predict the severity of depression.
Nevertheless, there are various sub-types of depression.
For instance, pathogenesis and functioning differ between bipolar disorder and the most prevalent major depressive disorder~\cite{konarski2008volumetric}.
There has not been much study on how the two differ using depression data up to this point.

\subsubsection{Longitudinal study}


The majority of current research has relied on cross-sectional controlled studies and has not carried out multi-temporal, whole-course follow-up studies of depression prior to onset, during onset, and during remission.
Determining the neuropathological mechanisms underlying this illness requires thoroughly examining the entire course of depression.
In upcoming research, long-term follow-up studies of depressed patients will be more beneficial in understanding the pathological mechanisms during the onset and progression of depression while defining the pathological changes in different disease stages.

%Depression can be further subdivided to explore and analyze
%different types of depression.
%

%
%\subsubsection{Su  }
%
%
\subsection{Datasets}


Although the auxiliary depression diagnosis has progressed with existing datasets, there is still space for advancement in terms of universality, clinical, and publicity.

\subsubsection{Universality}
In the majority of past depression prediction studies, sample sizes have been relatively small.
A larger sample size is advantageous for constructing a more accurate prediction model that generalizes well to the entire population.
The following two factors must be considered to build a sizable and thorough depression database.
First, depression manifests differently by age, gender, cultural background, and disease state. The performance of classification studies can be improved by exploring large datasets covering patients of all age groups and genders.
Second, because the causes of depression are complicated, it is often necessary to collect and analyze multiple psychological and physiological indicators simultaneously to get an objective quantitative assessment of depression.
In future studies, we suggest collecting multimodal data such as brain imaging, EEG, and audio/video of subjects simultaneously to analyze them together.

%Currently, datasets employed in SDR research come normally from interactive clinical interview, in which the questions are carefully designed and there is no noise interference like in real life. Therefore, these data cannot fully reflect the normal life state of patients with
%depression. Besides, the issue of cross language and cultural has not yet been considered.



\subsubsection{Clinicality}

Research on machine learning-based auxiliary depression diagnosis must first understand the medical mechanism of depression.
It is necessary to gather and extract clinical information about depression to increase recognition accuracy.
Therefore, a more thorough examination of hospitalized patients is required to understand the diagnosis and course of the disease.
Additionally, follow-up research should improve interaction and coordination with other relevant professionals.
It is an essential and ongoing topic to investigate the mechanisms of depression in patients.


\subsubsection{Publicity}
Due to the sensitivity of depression data and ethical issues, most institutions cannot access adequate samples.
There are few publicly available databases on depression, which leaves a large gap in scientific research needs.
Future studies' sample sizes might grow if there is an open-access repository with international contributors.
In this manner, medical experts and researchers worldwide could collaborate to produce much larger datasets.
This would subsequently impact the test and training set, increasing the accuracy and reliability of any model eventually produced to move forward with a clinical trial.




\subsection{Model}
Many factors affect the model performance, such as unbalanced samples, model interpretability, applicability, and multimodal combination. For these problems, we give some insights into the solutions.

\subsubsection{Unbalance samples}
Sample imbalance is prevalent in studies of auxiliary depression diagnosis, especially in deep learning approaches driven by data. We can use data augmentation and adversarial generation methods to equalize the training data for this problem. Or train on other datasets and then migrate to the current data. Further, it is possible to provide different weights for different categories of samples and set larger weights for classes with fewer samples.

\subsubsection{Interpretability}
Deep learning-based approaches to depression diagnosis usually achieve better results than traditional machine learning methods, but it is less explanatory. Combining features extracted by manual and neural networks for analysis and prediction can improve model performance and provide interpretability.

\subsubsection{Applicability}
Deep learning-based methods usually require the manual design of model architectures, which may not be the best fit for the data. So, we can use neural architecture search to find the best neural network architecture to produce good models for a given dataset.

\subsubsection{Multimodal fusion}
Single modality does not reflect depression features comprehensively enough. At the same time, multimodal data can broaden the information coverage of input data with the help of complementary information, thus improving the correct rate of depression prediction. Further optimization of the feature space and exploration of the coupling relationship between multimodal data is needed in future research to construct models with high classification accuracy and better robustness.

\subsection{Application}

We propose several future research directions for auxiliary depression diagnostic methods in terms of applied innovation:

\subsubsection{Clinical Application}
In clinical diagnosis, most people ignore depression, and the misconception that depressed patients are disease-free leads to untimely diagnosis, resulting in patients not receiving timely treatment. Therefore, data-based behavioral depression diagnostic methods can be introduced to assist in diagnosing patients.
Secondly, auxiliary depression diagnostic methods can be used to monitor the physical changes of depressed patients for a long time and assess the effects of medications or other factors on the recovery of depressed patients to provide more help in finding effective treatments.

\subsubsection{Daily Monitor}
Depression is often accompanied by abnormal emotions and is a common mental illness. With the outbreak of the epidemic, there has been a dramatic increase in depression across all age groups. Daily monitoring using easily collected behavioral data provides real-time insight into one's condition and allows prompt medical attention when abnormalities are detected.
Secondly, mass screening can be conducted in colleges, communities, and other crowded areas using complementary depression diagnostic methods to diagnose people who may be suffering from depression promptly.



\subsection{Conclusion}
%%%%%%%%%%%%%%%%%%%%%%%%%%%%%%%%%%%%%%%


Depression is the fourth most significant disease in the world, but the medical prevention and treatment of depression are still in a situation of low recognition rate.
It is significant for academic research and clinical care to develop an automatic and objective evaluation system.
This paper provides a systematic analysis of the literature from the perspective of machine learning applied to different modal data, summarizes the general research process and typical research methods in machine learning for aided diagnosis of depression, and looks at future research directions. It will help psychiatric researchers in creating more dependable and computationally intelligent systems.



\ifx\allfiles\undefined
% !TEX root = tnnls_relation_gait.tex

% if have a single appendix:
%\appendix[Proof of the Zonklar Equations]
% or
%\appendix  % for no appendix heading
% do not use \section anymore after \appendix, only \section*
% is possibly needed

% use appendices with more than one appendix
% then use \section to start each appendix
% you must declare a \section before using any
% \subsection or using \label (\appendices by itself
% starts a section numbered zero.)
%

%\appendices
%\section{Proof of the First Zonklar Equation}
%Appendix one text goes here.
%
%% you can choose not to have a title for an appendix
%% if you want by leaving the argument blank
%\section{}
%Appendix two text goes here.

% use section* for acknowledgment
% \section*{Acknowledgment}
% The authors would like to thank Prof. Dongbin Zhao for his support to this work.

% Can use something like this to put references on a page
% by themselves when using endfloat and the captionsoff option.
\ifCLASSOPTIONcaptionsoff
  \newpage
\fi

% trigger a \newpage just before the given reference
% number - used to balance the columns on the last page
% adjust value as needed - may need to be readjusted if
% the document is modified later
%\IEEEtriggeratref{8}
% The "triggered" command can be changed if desired:
%\IEEEtriggercmd{\enlargethispage{-5in}}

% references section

% can use a bibliography generated by BibTeX as a .bbl file
% BibTeX documentation can be easily obtained at:
% http://mirror.ctan.org/biblio/bibtex/contrib/dsoc/
% The IEEEtran BibTeX style support page is at:
% http://www.michaelshell.org/tex/ieeetran/bibtex/
\bibliographystyle{IEEEtran}
% argument is your BibTeX string definitions and bibliography database(s)
\bibliography{IEEEabrv,tnnls_relation_gait}
% \bibliography{IEEEabrv,1}
%
% <OR> manually copy in the resultant .bbl file
% set second argument of \begin to the number of references
% (used to reserve space for the reference number labels box)
%\begin{thebibliography}{1}
%\bibitem{IEEEhowto:kopka}
%H.~Kopka and P.~W. Daly, \emph{A Guide to \LaTeX}, 3rd~ed.\hskip 1em plus
%  0.5em minus 0.4em\relax Harlow, England: Addison-Wesley, 1999.
%\end{thebibliography}

% biography section
%
% If you have an EPS/PDF photo (graphicx package needed) extra braces are
% needed around the contents of the optional argument to biography to prevent
% the LaTeX parser from getting confused when it sees the complicated
% \includegraphics command within an optional argument. (You could create
% your own custom macro containing the \includegraphics command to make things
% simpler here.)
%\begin{IEEEbiography}[{\includegraphics[width=1in,height=1.25in,clip,keepaspectratio]{mshell}}]{Michael Shell}
% or if you just want to reserve a space for a photo:

%\begin{IEEEbiography}{Michael Shell}
%Biography text here.
%\end{IEEEbiography}
%
%% if you will not have a photo at all:
%\begin{IEEEbiographynophoto}{John Doe}
%Biography text here.
%\end{IEEEbiographynophoto}

% insert where needed to balance the two columns on the last page with
% biographies
% \newpage

%\begin{IEEEbiographynophoto}{Jane Doe}
%Biography text here.
%\end{IEEEbiographynophoto}


% You can push biographies down or up by placing
% a \vfill before or after them. The appropriate
% use of \vfill depends on what kind of text is
% on the last page and whether or not the columns
% are being equalized.

%\vfill

% Can be used to pull up biographies so that the bottom of the last one
% is flush with the other column.
%\enlargethispage{-5in}

% that's all folks
\end{document} 
\fi
