% !TEX root = tnnls_depression_survey.tex

\ifx\allfiles\undefined
    \input{tnnls_prefix}
\fi

%\section{Depression Recognition}
\section{Auxiliary depression diagnosis method based on electrophysiological signals}
%%%%%%%%%%%%%%%%%%%%%%%%%%%%%%%%%%%%%%%%%%%%%%%%%%%%%%%

Depression as a psychiatric disorder is frequently accompanied by abnormal brain activity and major mood alternations.
EEG, as a method of tracking brain function, can detect these abnormal activities.
Further, depressive episodes may be accompanied by nausea, vomiting, chest tightness, and sweating.
All of these symptoms are associated with autonomic dysfunction, which can be substantially reflected by signals such as the ECG, heart sounds, and pulse.
Specifically, analysis of heart rate variability (HRV) is used extensively as a standard tool for evaluating autonomic nervous functions.
Moreover, research indicates that when the body is exposed to external stimuli or changes in mental activity, electrodermal activity is more responsive than electrocardiographic or electroencephalographic activity and generates more pronounced alterations in reaction to stimuli.
Besides that, breathing plays a crucial role in the regulation of autonomic systems.
In summary, the electrophysiology of depressed patients differs from that of healthy individuals.



%The electrophysiology of depression includes EEG, ECG, electrodermal, gastric, ophthalmic, and body temperature, and they are usually recorded by metal electrodes placed on the body surface.

Electrophysiological signals are caused by changes in the membrane potential of individual cells, which are usually recorded by metal electrodes placed on the body surface.
Due to the non-invasive detection and simple use of electrophysiological signals, auxiliary depression diagnosis with them has become a hot research topic and the most promising research direction.
Actually, related researches are shifting from early traditional methods based on hand-crafted features to the implementation of deep learning architectures.
The methodological flow is as follows.
(1)Traditional machine learning: it begins with feature extraction, then feature selection to acquire more effective and representative aspects of depression, and then creating prediction or clustering models based on these features to perform the classification problem.
(2)Deep Learning: it sends the original signal to a deep network so that it may automatically learn high-level features.
Since it is able to address the issues experienced with human-crafted features, such as high threshold, labor expense, and low feature utilization rate, deep learning is gradually gaining the upper hand in the field of machine learning.




Depression's electrophysiology covers EEG, ECG, electrodermal, gastric, EOG, and body temperature measurements. EEG is utilized the most commonly~\cite{liao2017major,bairy2017automated,acharya2018automated}, followed by ECG~\cite{ding2019classifying}. It is noteworthy to note that ECGs are commonly used to diagnose depression by extracting RR interval sequences and measuring HRV. In addition, as other types of signals are employed infrequently, we concentrate on EEG and HRV-based auxiliary depression diagnosis.

%\begin{figure}
%\centering
%\includegraphics[height=5cm,width=8cm]{EEG-HRV.jpg}
%\caption{Examples of electrophysiological signals.}
%\label{signals01}
%\end{figure}

\subsection{EEG}

EEG is the most direct and objective representation of neurological brain activity in humans.
It has a close relationship with brain activity and emotional state, provides a timely representation of the body's emotional changes, and offers a high temporal resolution for examining brain dynamics.
Therefore, the EEG is one of the most promising signals that may be utilized for auxiliary depression diagnosis.
Current research continues to rely mostly on conventional machine learning methods.
In the meantime, a small number of academics have experimented with a subset of machine learning known as deep learning algorithms, which has led to significant advances.


\subsubsection{EEG auxiliary depression diagnosis based on traditional machine learning}
%The main points of a traditional machine learning-based method can probably be summarized as feature extraction and classification.
%The feature extracted from the EEG signals can widely be divided into two categories: linear analysis and non-linear analysis,
%the linear analysis can be further classified into two sub-categories: time domain, frequency domain analysis.

Feature extraction and classification are arguably the two key components of a traditional machine learning approach.
The derived features from EEG signals can be broadly grouped into two categories: linear analysis and non-linear analysis, with linear analysis further subdivided into time domain and frequency domain analysis.




Typically, time domain analysis focuses on the measurement and analysis of the raw EEG signal waveform in order to calculate more intuitive or physiologically significant EEG properties.
In the study of EEG auxiliary depression diagnosis, the time domain features generally used include the mean, variance, kurtosis, and skewness of the signal.
These features vary with the EEG signal's sliding window and precisely depict their temporal evolution.
Additionally, time-domain analysis is able to measure the synchronization and similarity of signals between various channels.
In general, time-domain features of EEG signals are very straightforward and intuitive characteristics with evident physical significance.
However, basic linear indicators may not be adequate for describing the dynamic activities of complex interactions, and it is challenging to obtain a complete picture of the EEG by examining it just in the temporal domain.
%However, due to its limitations, it is challenging to provide a thorough picture of EEG using only temporal domain analysis.




Frequency domain analysis refers to the process of characterizing a non-stationary EEG signal based on its frequency components and calculating all important properties at the frequency.
The most often employed frequency domain characteristic is the power spectral density, which is normalized to estimate the power of the time series at each frequency.
In addition, typical methods for characterizing EEG signals in the frequency domain include bispectral analysis and spectral estimation based on AR parameter models.
Although power spectral density may evaluate the information of the frequency variation of EEG signals, it cannot capture the time-varying information at the frequency components of interest simultaneously for EEG signals with significant unpredictability.



Non-linear approaches are able to capture the chaotic behavior and abrupt changes in EEG signal caused by a brain-based physiological phenomena (Rodreguez-Bermudez and Garcia-Laencina 2015).
In EEG auxiliary depression diagnosis research, prominent nonlinear features include correlation dimension, Lyapunov exponent, entropy, Higuchi's fractal dimension, relative wavelet energy, and detrended fluctuation analysis, among others.
Each nonlinear indicator reflects the peculiar qualities of EEG from a different perspective, revealing information that may not be detected by other methods.
For instance, the correlation dimension indicates the system's complexity; the greatest Lyapunov exponent reflects the rate of uncertainty expansion and is sensitive to the system's initial value; and the entropy quantifies the system's amount of order.
On this basis, the researchers concurred that a combination of numerous nonlinear approaches is superior to a single method.
To verify the aforementioned conclusion, Acharya et al.~\cite{acharya2015novel} present a novel method for automated auxiliary depression diagnosis based on non-linear techniques: fractal dimension, largest Lyapunov exponent, sample entropy, detrended fluctuation analysis, Hurst's exponent, higher order spectra, and recurrence quantification analysis.
The results suggest that by judiciously combining non-linear features, it may objectively and effectively auxiliary depression diagnosis.




%For EEG auxiliary depression diagnosis studies, numerous non-linear parameters have been discovered, including the correlation dimension \cite{akar2015nonlinear}, the Lyapunov exponent\cite{ortiz2021futility,kustubayeva2021lyapunov,chopra2022using}, and entropy \cite{zhao2019cardiorespiratory,cukic2018eeg,zhao2021frontal}.
%The correlation dimension describes the system's complexity;
%the largest Lyapunov exponent reflects the pace of uncertainty growth and is sensitive to the system's beginning value; and the entropy depicts the level of order in the system.
%%%%%%%%%%%%%%%%%%%%%%%%%%%%%%%%%%%%%%%%%%%%%%%%%%%%%%%%%%%%%%%%%%%%%%%%%%%%%%%%%%%%%%%%%%%%%

%There are two key points to consider based on traditional machine learning methods: feature extraction and classifiers.
%Further, according to the characteristics of EEG signals, feature extraction can be roughly summarized into two categories: linear and non-linear.
%However, brain activity is inherently non-linear, and the EEG signal is non-stationary \cite{walleczek2006self,Scott2006encyclopedia,klonowski2006application}, which lead to limitations in the linear approach to assessing the dynamics of depressive EEG.
%In contrast, the non-linear approach demonstrates a relatively clear advantage.
%Therefore, here we highlight the non-linear characteristics of the EEG signal.
%%%%%%%%%%%%%%%%%%%%%%%%%%%%%%%%%%%%%%%%%%%%%%%%%%%%%%%%%%%%%%%%%%%%%%%%%%%%%%%%%%%%%%%%%%%%%%%%%%%%%%%%

%The main process of EEG depression identification using traditional machine learning is shown in Fig~\ref{eeg_traditional}.
%There are two key points to emphasize: feature extraction and classifiers.
%According to the characteristics of EEG signals, feature extraction can be roughly  further summarized into two categories: linear and non-linear.
%However, brain activity is inherently non-linear, and the EEG signal is non-stationary \cite{walleczek2006self,Scott2006encyclopedia,klonowski2006application}, which lead to limitations in the linear approach to assessing the dynamics of depressive EEG.
%In contrast, the non-linear approach demonstrates a relatively clear advantage.


%The non-linear features commonly used in depression EEG studies include correlation dimension, Lyapunov exponent, and entropy. Among them,

%The non-linear dynamics theory suggests that a self-organizing, low-dimensional, deterministic system can generate complicated and non-periodic behavior that is inherent to the system rather than generated from external sources.

%%%%%%%%%%%%%%%%%%%%%%%%%%%%%%%%%%%%%%%%%%%%%%%%%%%%%%%%%%%%%%%%%%%%%%%%%%%%%%%%%%%%%%%%%%%%%%%%%%%%%%%%%%%%%%%
%%%%%%%%%%%%%%%%%%%%%%%%%%%%%%%%%%% ԭ��%%%%%%%%%%%%%%%%%%%%%%%%%%%%%%%%%%%%%%%%%%%%%%%%%%%%%%%%
%For EEG auxiliary depression diagnosis studies, numerous non-linear parameters have been discovered, including the correlation dimension \cite{akar2015nonlinear}, the Lyapunov exponent\cite{ortiz2021futility,kustubayeva2021lyapunov,chopra2022using}, and entropy \cite{zhao2019cardiorespiratory,cukic2018eeg,zhao2021frontal}.
%The correlation dimension describes the system's complexity;
%the largest Lyapunov exponent reflects the pace of uncertainty growth and is sensitive to the system's beginning value; and the entropy depicts the level of order in the system.
%Researchers also employ a combination of multiple non-linear indictors because they can each reflect the distinctive properties of EEG from a different angle, exposing information that may not be picked up by other approaches.
%For example, Acharya et al.\cite{acharya2015novel} presents a novel method for automated auxiliary depression diagnosis using nonlinear methods: fractal dimension, largest Lyapunov exponent, sample entropy, detrended fluctuation analysis, Hurst's exponent, higher order spectra, and recurrence quantification analysis.
%The results suggest that by judicious combination of the nonlinear features, it is conceivable to objectively and effectively auxiliary depression diagnosis.
%%%%%%%%%%%%%%%%%%%%%%%%%%%%%%%%%%%%%%%%%%%%%%%%%%%%%%%%%%%%%%%%%%%%%%%%%%%%%%%%%%%%%%%%%%%%%%%%%%%%%%
%%%%%%%%%%%%%%%%%%%%%%%%%%%%%%%%%%%%%%%%%%%%%%%%%%%%%%%%%%%%%%%%%%%%%%%%%%%%%%%%%%%%%%%%%%%%%%%%%%%%%




%Although the indicators employed differ, the prevalent classifiers are typically the same: k-Nearest Neighbor (KNN), linear discriminant analysis, linear regression (LR), support vector machines (SVM), decision trees, Bayesian classifiers, random forests, and probabilistic neural networks.
%Further, Mahato et al.~\cite{mahato2019electroencephalogram} compares various EEG signal analysis techniques and recommends the most suitable technique based on the accuracy for detection of depression.
%The study reveals that, in general, high classification accuracy is achieved by SVM, LR and ANN and highest classification accuracy of 98.33\% is achieved by SVM. Highest accuracy is achieved by SVM because it is more robust and computationally more efficient due to maximal margin gap between separating hyper planes and kernel trick.

Even though the applied indicators vary, the most prevalent classifiers are often the same: k-Nearest Neighbor (KNN), linear discriminant analysis, linear regression (LR), SVM, decision trees, Bayesian classifiers, random forests, and probabilistic neural networks.
In addition, Mahato et al.~\cite{mahato2019electroencephalogram} examine numerous EEG signal analysis approaches and recommend the best accurate technique for auxiliary depression diagnosis.
It is demonstrated that SVM has the highest classification accuracy compared to other approaches. Due to the fact that SVM has the greatest distance between the separation hyperplane and the kernel trick, it is more robust and computationally efficient, resulting in the greatest accuracy.


%Hosseinifard et al. \cite{hosseinifard2013classifying} used associative dimensional features as input to a linear regression classifier and obtained 90\% recognition accuracy.
%Ahmadlou et al.\cite{ahmadlou2012fractality} fed fractal dimensional features of depressed patients into an augmented probabilistic neural network and obtained an accuracy of 91.3\%.
%Cukic et al. \cite{cukic2018eeg} obtained an accuracy of 90.24\%-97.56\% by feeding Higuchi fractal dimension and sample entropy into seven classifiers.


%As stated above, researchers use different physiological signals and features in physiological signal-based depression identification, which leads to slightly different outcomes.
%However, the the common classifiers are all consistent, including SVM, k-Nearest Neighbor (KNN), Logistic Regression (LR), and Random Forest (RF).

\subsubsection{EEG auxiliary depression diagnosis based on deep learning}
%\subsection{Application of deep learning in depression recognition}
%Applications of deep learning in depression diagnosis with the assistance of physiological signals have increased in recent times.
%All adopted models can be categorized into three main groups, namely Convolutional Neural Network (CNN)-Based models, Long-short Term Memory (LSTM) models, a hybrid neural network.
In recent years, deep learning has been utilized more and more to detect depression with the help of EEG.
All adopted models in these studies can be categorized into three main groups, namely Convolutional Neural Network (CNN) models\cite{wan2020hybrideegnet}, Long-short Term Memory (LSTM) models, and hybrid neural networks\cite{saeedi2021major,qayyum2020hybrid,thoduparambil2020eeg}. And their detailed flow is shown in Fig~\ref{eeg_cnn}.

\begin{figure}
\centering
\includegraphics[width=1\linewidth]{figures/depression/eeg_cnn.jpg}
\caption{Flow diagram of EEG depression recognition based on deep learning. (a) Raw signal-based deep feature extraction and classification using CNN or LSTM; (b) raw signal-based deep feature extraction and classification using hybrid neural networks; (c) 2D image-based deep feature extraction and classification using hybrid neural networks. }
\label{eeg_cnn}
\end{figure}

In specific, %CNN could capture spatial properties of features and has the ability of parallel computing.
the CNN-based model can automatically and adaptively extract useful information directly from the input data instead of manually selecting features.
%Additionally, it has two significant groups of data input formats: time series and feature images.
Acharya et al.\cite{acharya2018automated} presents the first application of the deep neural network concept and CNN for diagnosis of depression.
The model's inputs are the EEG signals from the left and right hemispheres of the brain, respectively.
5 convolutional layers are responsible for providing important feature
obtained from the input EEG signals to train the algorithm, 5 pooling layers are accountable for reducing of the feature map's size, and 3 fully-connected layers establish connection between neurons of a layer with the next one. Finally, results produces by the last fully-connected layer is applied to a soft-max function to detect depressive cases.
Similarly, Li et al.~\cite{li2019eeg} analyzes different aspects of EEG (spectral, spatial, and temporal information) for the diagnosis of of mild depression.
Results shows that spectral information of EEG signals play major role whereas temporal information shows significant improvement in diagnostic performance.
The uitlized the pre-trained ConvNet architectures on EEG-based mental load classification task and achieved accuracy of 85.62\% for recognition of mild depression and normal controls.
%Kang et al. \cite{kang2020deep} introduced a novel methodology of feature extraction to detect depression, which was exploiting asymmetry feature of EEG signals and converting it into 2D images.
%The produced image used as input of a CNN model which had constructed from three two-dimensional
%convolution layers with the ReLU as activation function, three two-dimensional max-pooling layers, one flatten and one dropout layer, and two fully connected layers.


%As a model capable of remembering long-term dependent information, LSTM can read and modify the long-term dependent information of a temporal sequence arbitrarily when processing temporal signals like EEG.
The benefit of the LSTM when processing temporal signals such as EEG is that it has the ability to read and modify the long-term dependent information of a temporal sequence at will\cite{alhagry2017emotion,wang2018lstm}.
Only a little portion of LSTM research, nevertheless, have focused on detecting depression through physiological signals.
Kumar et al. \cite{kumar2019prediction} research depression prediction by the LSTM model and the help of feature extraction.
To generate input data, time-domain analysis with moving window segmentation is employed to extract the statistical mean feature.
LSTM block comprises of one LSTM layer with ten hidden neurons, a dropout layer of 0.1, and a dense layer.
They also compare the introduced method with two other models that are CNN-LSTM and ConvLSTM.
Among them, LSTM has the best performance over those structures since it owns the smallest RMSE values (Root mean square error) as model evaluator rather than them.


The majority of the hybrid neural network are combined models of both CNN and LSTM blocks.
Additionally, it has two significant groups of data input formats: time series and feature images.
Concretely, Thoduparambil et al.\cite{thoduparambil2020eeg} design a deep model in which an integration of CNN and LSTM is implemented for the detection of depression.
Three CNN layers and three MaxPooling1D layers form the first section of the model, which is exploited to extract features.
The second unit includes two LSTM layers whose duty is to generate the feature maps by discovering different patterns in EEG signals and then retain the sequence of these learning.
Similarly, Sharma et al. \cite{sharma2021dephnn} also propose a new hybrid neural network called DepHNN.
CNN and LSTM are two deep learning algorithms used to capture the temporal dependencies in the time-series EEG input signals and to process the sequence learning, respectively.
Besides, Saeedi et al.~\cite{saeedi2021major} use brain effective connectivity method to convert 1D EEG signal into 2D image. Then, they develop a classifier using state of the art deep learning methods (CNN-LSTM) as a novel approach for automated diagnosis of the depression patients from EEG signals.
%The effectiveness of the proposed approach is tested on dataset recorded of from 33 MDD patients and 30 normal participants.
Finally, their experiments shows that the spatial and temporal characteristics of the EEG
signals are captured by 1DCNN-LSTM. Relying on the results, deep learning model is also capable
of effectively analyzing the brain connectivity.


%Besides, Kang et al. \cite{kang2020deep} introduced a novel methodology of feature extraction to detect depression, which was exploiting asymmetry feature of EEG signals and converting it into 2D images.
%%The produced image used as input of a CNN model which had constructed from three two-dimensional
%%convolution layers with the ReLU as activation function, three two-dimensional max-pooling layers, one flatten and one dropout layer, and two fully connected layers.




\subsubsection{Performance Comparison}

We evaluate the EEG-based auxiliary depression diagnosis technique.
Table~\ref{tab_eeg} summarizes the performance comparisons between conventional machine learning and deep learning.
We would like to stress that we cannot compare all studies because the datasets used vary, however by comparing the work of the same scholars, the following conclusion can be drawn:

(1) Single-channel EEG analysis, employing the combination of measures, can provide the accuracy for discrimination of depression not lower than reported in other studies where multichannel EEG signals were analysed~\cite{bachmann2018methods}.

(2)
The nonlinear analysis of EEG can be a useful method for discriminating depressed patients and normal subjects.
According to \cite{hosseinifard2013classifying}, the accuracy of three classifiers are
higher for all nonlinear features as the input in compare to power bands features.
Brain system is best-characterized non-linear dynamical process.
The nonlinearity of brain limits the ability of linear analysis to provide full description of underlying dynamics.

(3)
In contrast, the novelty of the deep learning model is that it does not require the employment of feature extraction, selection, and reduction. The model has the ability to self-learn and pick up distinctive features during training without a separate feature extraction or feature selection step.
Acharya\cite{acharya2018automated} presented the first application of the deep neural network concept for diagnosis of depression.
Even though the accuracy is not as great as the previous one\cite{acharya2015novel}, deep learning technology development is anticipated to lead to new discoveries.

\begin{table*}[!htbp]\Large
\caption{ Overview of machine learning based methods for Depression
Assessment from EEG.}
\label{tab_eeg}
\renewcommand\arraystretch{1.5}
%\resizebox{\linewidth}{!}{
\resizebox{0.9999\linewidth}{!}{%
\begin{threeparttable}
\begin{tabular}{c|cccccc}
\toprule
\textbf{Methods} & \textbf{Paper} & \textbf{Dataset} & \textbf{Feature} & \textbf{Classification} & \textbf{Metrics}           & \textbf{Value} \\ \midrule
& &  & FD, LLX, & & & \\
& &  & Hurst,HOS, & & & \\
& &  & SampEn, DFA &  &  & \\
& \multirow{-4}{*}{Acharya\cite{acharya2015novel}}  & \multirow{-4}{*}{15D+15C} & and RQA                                                                     & \multirow{-4}{*}{SVM} & \multirow{-4}{*}{Accuracy} & \multirow{-4}{*}{98.00} \\ \cline{2-7}
& & & SAI, APV and RGP & &  & Single-channel:92.00 \\
& \multirow{-2}{*}{Bachmnn\cite{bachmann2018methods}} & \multirow{-2}{*}{13D+13C}  & HFD, DFA and Lempel-Ziv & \multirow{-2}{*}{LR} & \multirow{-2}{*}{Accuracy} & Multil-channel:90.00  \\ \cline{2-7}
& &  & Power & KNN &  & Linear/Nonlinear:73.3/80   \\
& & & DFA, HFD, CD  & LDA& & Linear/Nonlinear:76.6/86.6 \\
\textbf{EEG} & \multirow{-3}{*}{Hosseinifard\cite{hosseinifard2013classifying}} & \multirow{-3}{*}{45D+45C} & and lyapunov exponent & LR& \multirow{-3}{*}{Accuracy} & Linear/Nonlinear:76.6/90   \\ \cline{2-7}
\textbf{depression} & & && Multilayer perceptron &  & 95.12  \\
\textbf{recognition} & & & & LR &  & 97.56  \\
\textbf{based on} & &  & & SVM with linear kernel &  & 95.12  \\
\textbf{traditional} &  & &  & SVM with polynomial & & 95.12   \\
\textbf{machine}&  & &  & Decision tree  &  & 95.12  \\
\textbf{learning} &  & & & Random forest && 92.68  \\
& \multirow{-7}{*}{Cukic\cite{cukic2018eeg}}  & \multirow{-7}{*}{23D+20C} & \multirow{-7}{*}{HFD,SampEn}                                                              & Naive Bayes& \multirow{-7}{*}{Accuracy} & 92.68 \\ \cline{2-7}
& Ahmadlou\cite{ahmadlou2012fractality} & 12D+12C & FD & EPNN & Accuracy & 91.30 \\ \cline{2-7}
& Faust\cite{faust2014depression} & 30D+30C& ApEn, SampEn, REN and Ph & PNN& Accuracy & 99.7  \\ \cline{2-7}
& Liao\cite{liao2017major} & 12D+12C  & KEFBCS& SVM & Accuracy & 80.00 \\ \midrule
& Wan\cite{wan2020hybrideegnet}  & 23D+12C  & Raw signals  & HybridEEGNet& Accuracy & 79.08 \\ \cline{2-7}
& Saeedi\cite{saeedi2021major} & 34D+30C & Effective connectivity & 1DCNN-LSTM & Accuracy & 99.24 \\ \cline{2-7}
& Quayyam\cite{qayyum2020hybrid}  & & Raw signals & 1DCNN-GRU-RF & F1-score & 99.94 \\ \cline{2-7}
\textbf{EEG} & &  &  & & & Left hemisphere:98.84 \\
\textbf{depression}& \multirow{-2}{*}{Thoduparambil\cite{thoduparambil2020eeg}} & \multirow{-2}{*}{ } & \multirow{-2}{*}{Raw signals} & \multirow{-2}{*}{CNN-LSTM}  & \multirow{-2}{*}{Accuracy} & Right hemisphere:99.07     \\ \cline{2-7}
\textbf{recognition} & & &  &  & & Left hemisphere:93.5       \\
\textbf{based on} & \multirow{-2}{*}{Acharya\cite{acharya2018automated}} & \multirow{-2}{*}{15D+15C} & \multirow{-2}{*}{Raw signals} & \multirow{-2}{*}{CNN} & \multirow{-2}{*}{Accuracy} & Right hemisphere:96.0      \\ \cline{2-7}
\textbf{deep} & Kang\cite{kang2020deep} & 34D+30C  & Asymmetry matrix image & CNN  & Accuracy & 98.85 \\  \cline{2-7}
\textbf{learning}& Kumar\cite{kumar2019prediction} & 30D & time domain feature& LSTM  & RMSE  & 0.005 \\ \cline{2-7}
& Sharma\cite{sharma2021dephnn} & 21D+24C & Raw signals & DepHNN & Accuracy & 99.1  \\ \cline{2-7}
& Li\cite{li2019eeg} & 24D+24C& spectral, spatial, and temporal information & ConvNet & Accuracy & 85.62  \\ \cline{2-7}
& &  &  &  &  & Left hemisphere:99.12      \\
& \multirow{-2}{*}{Ay\cite{ay2019automated}} & \multirow{-2}{*}{15D+15C} & \multirow{-2}{*}{Raw signals}                                                                              & \multirow{-2}{*}{CNN-LSTM}  & \multirow{-2}{*}{Accuracy} & Right hemisphere:97.66\\ \bottomrule
\end{tabular}
      \begin{tablenotes} %���Ӵ˴�
		\item Largest Lyapunov exponent: LLX; Higher order spectra: HOS; Recurrence quantification analysis: RQA;
              Spectral asymmetry index: SAI;
        \item Alpha power variability: APV; Relative gamma power: RGP; Detrended fluctuation analysis: DFA; Linear discriminant analysis: LDA; Sample Entropy: SampEn; % ���Ӵ˴�
        Fractal Dimension: FD; Enhanced probabilistic neural network: EPNN;
        approximate entropy: ApEn; Renyi entropy: REN; Bispectral phase entropy: Ph; Kernel eigen-filter-bank common spatial pattern: KEFBCS.
     \end{tablenotes} % ���Ӵ˴�
\end{threeparttable}}
\end{table*}


%%%%%%%%%%%%%%%%%%%%%%%%%%%%%%%%%%%%%%%%%%%%%%%%%%%%%%%%%%%%%%%%%%%%%%%%%%%%%%%%%%
%%%%%%%%%%%%%%%%%%%%%%%%%%%%%%%%%%%%%%%%%%%%%%%%%%%%%%%%%%%%%%%%%%%%%%%%%%%%%%%%
% Please add the following required packages to your document preamble:
% \usepackage{booktabs}
% \usepackage{multirow}
% \usepackage[table,xcdraw]{xcolor}
% If you use beamer only pass "xcolor=table" option, i.e. \documentclass[xcolor=table]{beamer}
% Please add the following required packages to your document preamble:
% \usepackage{booktabs}


\subsection{ECG}

Physiological variation of the interval between consecutive heartbeats is known as HRV. HRV analysis is traditionally performed on ECG signals and has become a useful tool in auxiliary depression diagnosis~\cite{heart1996standards}.
Studies have shown that people with psychiatric disorders such as depression have reduced HRV, as evidenced by reduced parasympathetic control and increased sympathetic activity~\cite{jandackova2016heart}.
In recent study, there are two prevalent strategies for HRV-based  auxiliary depression diagnosis: traditional methods with  hand-crafted features and end-to-end deep learning methods.
%Current, most studies have used conventional machine learning methods combined with HRV to auxiliary depression diagnosis. Although these methods achieve favourable classification performance, they have many drawbacks.





\subsubsection{HRV auxiliary depression diagnosis based on traditional machine learning}

Traditional machine learning, a mainstream paradigm in HRV auxiliary depression diagnosis, tries to discover depression-related variables and expand the feature set to improve performance.
Specifically, the feature set consists of time-domain HRV features, frequency-domain, and nonlinear features.
%In the interim,traditional machine learning algorithms such as SVM, Logistic Regression, Linear Discriminant Analysis, etc are employed to do the auxiliary diagnosis.




%%%%%%%%%%%%%%%%%%%%%%%%%%%%%%%%%%%%%%%%%%%%%%%%%%%%%%%%%%%%%%%%%%%%%%%%%%%%%%%%%%%%%%%%%%
%%%%%%%%%%%%%%%%%%%%%%%%%%%%%%%%%%%%%%%%%%%%%%%%%%%%%%%%%%%%%%%%%%%%%%%%%%%%%%%%%%%%%%%%%%
Time domain analysis of HRV is used to assess the linear activity state of the cardiovascular system. Common time-domain features were calculated based on the RR intervals using time-domain analysis, standard deviation of RRIs (SDNN), the root mean square of successive RR interval differences (RMSSD), standard deviation of the successive difference of RR intervals (SDSD), percentage of intervals greater than 50 ms (PNN50), and mean value of RR intervals (MEAN). SDNN reflects both sympathetic and parasympathetic activities; RMSSD, SDSD, and PNN50 are sensitive to parasympathetic modulation; and MEAN is the mean of RR intervals associated with the HRV intension.
%%%%%%%%%%%%%%%%%%%%%%%%%%%%%%%%%%%%%%%%%%%%%%%%%%%%%%%%%%%%%%%%%%%%%%%%%%%%%%%%%%%%%%%%%%%%%%%%%%%%
%%%%%%%%%%%%%%%%%%%%%%%%%%%%%%%%%%%%%%%%%%%%%%%%%%%%%%%%%%%%%%%%%%%%%%%%%%%%%%%%%%%%%%%%%%%%%%%%


%Various statistical methods are employed to objectively describe the variability of the cardiac cycle using HRV time domain analysis.
%Time-domain HRV characteristics can be determined directly from the R-peak to R-peak interval time series (RRI).
%Common time domain features are the mean and standard deviation of RRIs (SDNN) and the root mean square of successive RR interval differences (RMSSD).
%SDNN is the most intuitive metric of overall HRV size and is frequently used to quantify cardiovascular adaptations; it is considerably lower in depressed individuals compared to healthy participants. RMSSD indicates the fast-varying component of HRV and can be used as a measure of the parasympathetic modulation of heart rate; it is much lower in depressed individuals compared to normal persons.
%
%The time domain analysis of HRV is utilized to objectively quantify the heart cycle variability with a variety of statistical methodologies.
%Time-domain HRV characteristics can be determined directly from the R-peak to R-peak interval time series (RRI).
%Common time domain features are the mean and standard deviation of RRIs (SDNN) and the root mean square of successive RR interval differences (RMSSD).
%SDNN is the most intuitive metric of overall HRV size and is frequently used to quantify cardiovascular adaptations; it is considerably lower in depressed individuals compared to healthy participants. RMSSD indicates the fast-varying component of HRV and can be used as a measure of the parasympathetic modulation of heart rate; it is much lower in depressed individuals compared to the normal person
%
%
%
%Time domain analysis of HRV is used to quantitatively characterize the variability of the cardiac cycle by various statistical methods.
%Time-domain HRV features can be calculated directly from the time series of R-peak to R-peak interval (RRI)\cite{sgoifo2015autonomic}.
%The mean and standard deviation of RRIs (SDNN)\cite{shaffer2017overview}, and the root mean square of successive RR interval differences (RMSSD) are all commonly used time domain features\cite{schneider2017validity}.
%SDNN is the most intuitive measure of overall HRV size and is commonly used to measure cardiovascular adaptations, and is significantly lower in depressed patients than in normal subjects RMSSD reflects the fast-varying component of HRV and can be used as a measure of the magnitude of parasympathetic regulation of heart rate, and is significantly lower in depressed patients than in normal subjects.
%%Whereas the mean of RRIs measures the intensity of HRV \cite{colombo2015clinical}, the SDNN measures overall HRV and reflects both sympathetic and parasympathetic activity in the autonomic nervous system\cite{byun2019detection}, and the RMSSD and PNN50 are more sensitive to parasympathetic modulation.


Frequency-domain provides an assessment of vagal modulation of the RRI, extracted from the ECG.
It is mostly commonly acquired by fast Fourier transformation (FFT).
The RRI can be separated into three components depending on the frequency band: very low frequency (VLF, 0.003-0.04 Hz), low frequency (LF, 0.04-0.15 Hz), and high frequency (HF, 0.15-0.4 Hz) band \cite{kidwell2018heart}. According to previous studies, HF is primarily affected by parasympathetic activities while LF is influenced by both sympathetic and parasympathetic activity.
The ratio of LF to HF indicates that sympathetic activity predominates as compared to parasympathetic activity\cite{billman2013lf}.
For the changes in HRV frequency domain parameters in depressed patients, there are more consistent results both nationally and internationally, with significantly lower LF in depressed patients and a significant positive correlation between depressed mood and LF/HF.




%The typical HRV markers mentioned above are frequently used to detect depression as well.
%Kemp~\cite{kemp2012depression} summarized the achievements of numerous researchers and discovered that depressive individuals had much lower SDNN and RMSDD values in the time domain, HF values in the frequency domain, and HF/LF values in the frequency domain than healthy controls.
%The differences in HRV between individuals with severe depression and healthy controls have been further investigated in numerous subsequent research.
%Additionally to these markers, it was discovered that compared to healthy participants, depressive patients exhibited considerably lower PNN50 in the temporal domain and greater LF in the frequency domain~\cite{shinba2014altered}.

%%As heart rate fluctuation is a complex behavior originating from nonlinear regulatory processes, adaptation of nonlinear dynamics and information theory for HRV analysis has been suggested.
%Although HRV has been analyzed traditionally using linear methods, such as time-domain and frequency-domain analyses, growing evidence has demonstrated that linear HRV measures may not correctly represent the complex dynamics of heartbeat regulation modulated by the ANS~\cite{goldberger2002fractal}and that linear HRV features show a relatively higher inter-subject variability than nonlinear HRV indices, suggesting the importance of nonlinear HRV analysis.
%Schulz et al.~\cite{schulz2010altered} have shown that nonlinear HRV indices allow more reliable discrimination of major depressive disorder (MDD) patients from controls than linear HRV features, as the latter exhibit high inter-subject variability.
%In particular, the non-linear features used for EEG: fractal dimension, Lyapunov's index, entropy, etc., also apply to HRV.
%Greco et al. \cite{greco2018assessment} found that non-linear indicators such as fractal dimension, sample entropy and recursive graph analysis demonstrated significantly higher heart rate variability even in subclinical depression compared to healthy controls.
%Daniel et al. \cite{vigo2004relation}found that depressed patients over 60 years of age had significantly lower sample entropy values compared to healthy controls.



%%%%%%%%%%%%%%%%%%%%%%%%%%%%%%%%%%%%%%%%%%%%%%%%%%%%%%%%%%%%%%%%%%%%%%%%%%%%%%%%%%%%%%%
%%%%%%%%%%%%%%%%%%%%%%%%%%%%%%%%%%%%%%%%%%%%%%%%%%%%%%%%%%%%%%%%%%%%%%%%%%%%%%%%%%%%%%%
As heart rate fluctuation is a complex behavior originating from nonlinear regulatory processes, adaptation of nonlinear dynamics and information theory for HRV analysis has been suggested.
Commonly used metrics for complexity assessment of HRV sequences include: entropy, detrended fluctuation analysis, fractal dimension, lagged Panglais plot, quantitative recurrence analysis and Lyapunov's index. non-linear metrics such as fractal dimension, sample entropy and recurrence plot analysis were found by Greco et al. to demonstrate that even subclinical depressive states have significantly higher HRV complexity than healthy controls. Daniel et al. found that patients with depression over 60 years of age had significantly lower sample entropy values compared to healthy controls.



In order to recognize depression using HRV, a variety of classifiers can be used.
Kang et al. \cite{kuang2017depression} used the bayesian network to recognition depressed patients from a healthy people, the recognition results demonstrate the significant association between depression and HRV.
Byun et al.~\cite{byun2019detection} demonstrated the HRV-based diagnosis using SVM classifier. Monitoring the changes in linear and nonlinear HRV features for various autonomic nervous system states can facilitate the more objective identification of depression patients.


\subsubsection{HRV auxiliary depression diagnosis based on deep learning}
Different from conventional machine learning, the method based on deep learning can self-learn features from input signals without manual feature extraction and feature selection.
For HRV depression recognition based on deep learning, scholars build an end-to-end deep architecture and then push raw signal into deep architecture to let model learn high-level features by itself.
Specifically, they can be subdivided into two categories depending on the model used: CNN and hybrid models.


CNN dose not need to manually extract and select features. And, as an effective classifier, CNN can directly take ECG segments as input.
Concretely, Mohanraj et al.~\cite{mohanraj2022deep} propose system based on DL Neural Network (NN) called DesNN for detection of des pair level using ECG data of different subjects. The DesNN contains 7 conv layers, 5 bunch standard layers, 3 pooling layers, and 1 completely conn layers. The last completely associated layer utilizes softmax actuation work, while any remaining layers utilize cracked amended straight unit (LeakyReLU) enactment work.
Similarly, zang et al.~\cite{zang2022end} propose a end-to-end CNN network containing 2 convolution layers, 2 max-pooling layers and 1 fully-connected layer. The activation function of the convolutional layer is the rectified linear units (RELU). After each convolution layer, a max-pooling is applied to the obtained feature maps. The Softmax function of the fully-connected layer outputs normal or depression.
The experimental results indicate that the end-to-end deep learning approach can identify depression from ECG signals, and possess high diagnostic performance. It also shows that ECG is a potential biomarker in the diagnosis of depression.
%Although traditional machine learning techniques perform well, they nevertheless have several shortcomings, which are dependence on feature extraction and feature selection to a large extent.
%It indicates that the procedure consequently becomes both time and computationally-intensive.
%Unlike traditional machine learning, deep learning can automatically learn features from input data, therefore researchers are trying to use it to diagnose depression.


The majority of the hybrid neural network are combined models of both Recurrent Neural Network (RNN) and LSTM blocks.
The RNN model is a deep learning-based and easier than CNN.
Noor et al.~\cite{noor2021predicting} presented a model that used RNN and LSTM to predict the risk of depression based on HRV.
The RNN learns the signals from the dataset and passes those signals to the LSTM autoencoder. There are two parts to the autoencoder architecture in common. The encoder compresses the input, the decoder attempts to recreate it, and the recurrent autoencoder joins the encoder and decoder, capturing the regenerated ECG signals.
This model is much simpler than the other existing models, and the prediction results are praiseworthy.


%Noor et al.~\cite{noor2021predicting} presented a model that used Recurrent Neural Network (RNN)and LSTM to predict the risk of depression based on HRV.
%zang et al.~\cite{zang2022end} proposed a CNN network containing 2 convolutional layers, 2 max pooling layers and 1 fully connected layer to diagnose depression, and the results showed that the network achieved high classification performance.
%In conclusion, the end-to-end deep learning approach can identify depression from HRV signals, and possess high diagnostic performance.

%Although traditional approaches perform well in terms of classification, they also have significant flaws.
%For instance, HRV sequences are typically reflected in the ECG signal by variations in the RR interval.
%Nevertheless, various QRS wave detection techniques provide varying R points, which may have an impact on the classification outcomes of succeeding classifiers.
%Additionally, feature extraction and feature selection play a significant role in machine learning.
%This method requires a lot of computation in addition to time.
%Instead of requiring human feature extraction and feature selection, deep learning-based algorithms can learn features from the input signal on their own.
%For these reasons, researchers have tried to detect depression directly from ECG data using deep learning techniques.



\subsubsection{Performance Comparison}
%Table \ref{tab_ecg} summarizes the results of the experiments on the detection of depression based on HRV, including the source and name of the method, the data set used, the evaluation criteria of the experimental results.

Previous studies in which ECG/HRV features were used to discriminate depression patients have yielded promising results obtained via various machine learning methods, which are summarized in Table~\ref{tab_ecg}.
In these studies, Kuang et al.~\cite{kuang2017depression} achieved an accuracy of 86\% through selected HRV features, but only female subjects participated in this study; Zhang et al.~\cite{zhang2011new} used relatively small sample sizes. Traditional studies have used HRV features to discriminate patients with depression, and HRV is usually expressed from the RR interval collected from ECG data. It is inevitable that the QRS wave algorithm will be utilized to locate R-peak. The R-peak positioned by different QRS wave algorithms also has biases.
On the contrary, the deep learning method directly takes the ECG signal as input without extracting the HRV sequence.
In addition, traditional studies used simple classifiers to train features manually extracted from HRV sequences.
The performance of this type of classifier mainly depends on the feature selection process, which is laborious and time-consuming.
The deep model can overcome the shortcomings of manual feature extraction and selection.


%This research used data from 37 depressed patients and 37 normal controls. Small samples limit the use of other deep neural network models.


%Some previous studies [11, 12, 38] focused on classifying patients with major depressive disorder and healthy people, thus ignoring the identification of mild and moderate patients.


\begin{table*}[!htbp]\Large
\caption{ Overview of machine learning based methods for Depression
Assessment from HRV.}
\label{tab_ecg}
\renewcommand\arraystretch{1.5}
%\resizebox{\linewidth}{!}{
\resizebox{0.9999\linewidth}{!}{
\begin{threeparttable}
\begin{tabular}{c|cccccc}
\toprule
\textbf{Methods} & \textbf{Paper} & \textbf{Dataset}  & \textbf{Feature} & \textbf{Classification} & \textbf{Metrics}           & \textbf{Value}  \\ \midrule
   & Zhang\cite{zhang2011new} & 10D+10C & VLF,LF,HF,LF/HF,SDNN,RMSSD & Neuro-fuzzy network & Accuracy& 95.00 \\
&  & & RRI,SDNN,RMSSD,PNN50,TRI,TINN, & & &  \\
& & & VLF,LF,HF,LF/HF,Tot & \multirow{-2}{*}{SVM} & \multirow{-2}{*}{Accuracy} & \multirow{-2}{*}{74.4} \\
HRV depression recognition & &  & ApEn,SampEn,DFA,CD, & &  & \\
based on traditional  & \multirow{-4}{*}{Byuna\cite{byun2019detection}}    & \multirow{-4}{*}{31D+41C} & SD1,SD2 & \multirow{-2}{*}{Statistical filter} & \multirow{-2}{*}{Accuracy} & \multirow{-2}{*}{73.1} \\
machine learning & &  & HR,SDNN,PNN50, &  &  &  \\
& & & LF/HF,peakLF,peakHF,  &  &   &      \\
& \multirow{-3}{*}{Roh\cite{roh2014wearable}} & \multirow{-3}{*}{23D}  & LLE,SampEn,ApEn & \multirow{-3}{*}{SVM}                & \multirow{-3}{*}{Accuracy} & \multirow{-3}{*}{71}   \\
& Matsui\cite{matsui2016impaired}  & 13D+28C  & LF,HF  & LDA  & Accuracy  & 88  \\
& Sun\cite{sun2016objective} & 44D+47C & LF,HF,LF/HF  & Logistic regression & Accuracy & 79  \\
& & & SDNN,RMSSD,SDSD,PNN50,MEAN &  & &  \\
& & & VLF,LF,HF,  & &  &   \\
& \multirow{-3}{*}{Kuang\cite{kuang2017depression}} & \multirow{-3}{*}{38D+38C} & SampEn,DFA & \multirow{-3}{*}{Bayesian networks}  & \multirow{-3}{*}{Accuracy} & \multirow{-3}{*}{86}   \\  \midrule
HRV depression recognition & Noor\cite{noor2021predicting} & 5000D& Raw signals  & RNN    & Accuracy & 97.24 \\
recognition based on & zang\cite{zang2022end}   & 37D+37C & Raw signals& CNN & Accuracy & 93.96 \\
deep learning& Mohanraj\cite{mohanraj2022deep}  & 15D+15C & Raw signals & DesNN  & Accuracy & 90 \\ \bottomrule
\end{tabular}
      \begin{tablenotes}
		\item Standard deviation of the successive difference of RR intervals: SDSD; Mean value of RR intervals: MEAN;
      	Integral of the histogram of the RR interval divided by its height: TRI; Baseline width of the RR interval histogram:TINN; Total power: Tot; Standard deviation of the Poincar plot perpendicular to (SD1) and along (SD2) the line of identity: SD1, SD2.
     \end{tablenotes}
\end{threeparttable}}
\end{table*}

\subsection{Others}


\ifx\allfiles\undefined
% !TEX root = tnnls_relation_gait.tex

% if have a single appendix:
%\appendix[Proof of the Zonklar Equations]
% or
%\appendix  % for no appendix heading
% do not use \section anymore after \appendix, only \section*
% is possibly needed

% use appendices with more than one appendix
% then use \section to start each appendix
% you must declare a \section before using any
% \subsection or using \label (\appendices by itself
% starts a section numbered zero.)
%

%\appendices
%\section{Proof of the First Zonklar Equation}
%Appendix one text goes here.
%
%% you can choose not to have a title for an appendix
%% if you want by leaving the argument blank
%\section{}
%Appendix two text goes here.

% use section* for acknowledgment
% \section*{Acknowledgment}
% The authors would like to thank Prof. Dongbin Zhao for his support to this work.

% Can use something like this to put references on a page
% by themselves when using endfloat and the captionsoff option.
\ifCLASSOPTIONcaptionsoff
  \newpage
\fi

% trigger a \newpage just before the given reference
% number - used to balance the columns on the last page
% adjust value as needed - may need to be readjusted if
% the document is modified later
%\IEEEtriggeratref{8}
% The "triggered" command can be changed if desired:
%\IEEEtriggercmd{\enlargethispage{-5in}}

% references section

% can use a bibliography generated by BibTeX as a .bbl file
% BibTeX documentation can be easily obtained at:
% http://mirror.ctan.org/biblio/bibtex/contrib/dsoc/
% The IEEEtran BibTeX style support page is at:
% http://www.michaelshell.org/tex/ieeetran/bibtex/
\bibliographystyle{IEEEtran}
% argument is your BibTeX string definitions and bibliography database(s)
\bibliography{IEEEabrv,tnnls_relation_gait}
% \bibliography{IEEEabrv,1}
%
% <OR> manually copy in the resultant .bbl file
% set second argument of \begin to the number of references
% (used to reserve space for the reference number labels box)
%\begin{thebibliography}{1}
%\bibitem{IEEEhowto:kopka}
%H.~Kopka and P.~W. Daly, \emph{A Guide to \LaTeX}, 3rd~ed.\hskip 1em plus
%  0.5em minus 0.4em\relax Harlow, England: Addison-Wesley, 1999.
%\end{thebibliography}

% biography section
%
% If you have an EPS/PDF photo (graphicx package needed) extra braces are
% needed around the contents of the optional argument to biography to prevent
% the LaTeX parser from getting confused when it sees the complicated
% \includegraphics command within an optional argument. (You could create
% your own custom macro containing the \includegraphics command to make things
% simpler here.)
%\begin{IEEEbiography}[{\includegraphics[width=1in,height=1.25in,clip,keepaspectratio]{mshell}}]{Michael Shell}
% or if you just want to reserve a space for a photo:

%\begin{IEEEbiography}{Michael Shell}
%Biography text here.
%\end{IEEEbiography}
%
%% if you will not have a photo at all:
%\begin{IEEEbiographynophoto}{John Doe}
%Biography text here.
%\end{IEEEbiographynophoto}

% insert where needed to balance the two columns on the last page with
% biographies
% \newpage

%\begin{IEEEbiographynophoto}{Jane Doe}
%Biography text here.
%\end{IEEEbiographynophoto}


% You can push biographies down or up by placing
% a \vfill before or after them. The appropriate
% use of \vfill depends on what kind of text is
% on the last page and whether or not the columns
% are being equalized.

%\vfill

% Can be used to pull up biographies so that the bottom of the last one
% is flush with the other column.
%\enlargethispage{-5in}

% that's all folks
\end{document} 
\fi
